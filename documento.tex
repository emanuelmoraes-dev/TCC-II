% MODELO LATEX PARA TCC DO IFCE 
% ELABORADO POR CARINA TEIXEIRA DE OLIVEIRA
% SETEMBRO DE 2017

\documentclass[
12pt, % tamanho da fonte
openright, % capítulos começam em pág ímpar (insere página vazia caso preciso)
oneside, % para impressão somente frente. Oposto a twoside (frente e verso)
a4paper, % tamanho do papel. 
% 	% -- opções do pacote babel --
english,% idioma adicional para hifenização
%french,% idioma adicional para hifenização
%spanish,% idioma adicional para hifenização
brazil,	% o último idioma é o principal do documento
]{abntex2}

% ------------------------
% PACOTES
% Mapear caracteres especiais no PDF
\usepackage{cmap}
% Usa a fonte Latin Modern
\usepackage{times}			
% Selecao de codigos de fonte.
\usepackage[T1]{fontenc}		
% Codificacao do documento (conversão automática dos acentos)
\usepackage[utf8]{inputenc}		
% Indenta o primeiro parágrafo de cada seção.
\usepackage{indentfirst}		
% Controle das cores
\usepackage{color}
\usepackage{soulutf8}
% Inclusão de gráficos
\usepackage{graphicx}			
\usepackage{multicol}
\usepackage{multirow}
%capitulos
\usepackage{titlesec}
% Pacotes adicionais, usados apenas no âmbito do Modelo Canônico do abnteX2
% para geração de dummy text
\usepackage{lipsum}				
% Pacotes de citações
% Paginas com as citações na bibl
\usepackage[brazilian,hyperpageref]{backref}	 
% Citações padrão ABNT
\usepackage[alf]{abntex2cite}	
% Pacote para definir cor nas tabelas
\usepackage[table,xcdraw]{xcolor}
% Pacote usado para exibir blocos de códigos de programação
\usepackage{listings}
% Definindo as linguagens a serem exibidas com lstlisting
\definecolor{mediumgray}{rgb}{0.3, 0.4, 0.4}
\definecolor{mediumblue}{rgb}{0.0, 0.0, 0.8}
\definecolor{forestgreen}{rgb}{0.13, 0.55, 0.13}
\definecolor{darkviolet}{rgb}{0.58, 0.0, 0.83}
\definecolor{royalblue}{rgb}{0.25, 0.41, 0.88}
\definecolor{crimson}{rgb}{0.86, 0.8, 0.24}
\colorlet{punct}{red!60!black}
\definecolor{background}{HTML}{EEEEEE}
\definecolor{delim}{RGB}{20,105,176}
\colorlet{numb}{magenta!60!black}
\definecolor{json_base}{RGB}{0,50,0}
\definecolor{codegreen}{rgb}{0,0.6,0}
\definecolor{codegray}{rgb}{0.5,0.5,0.5}
\definecolor{codepurple}{HTML}{C42043}
\definecolor{backcolour}{HTML}{F2F2F2}
\definecolor{bookColor}{cmyk}{0,0,0,0.90}  
\color{bookColor}

\lstdefinelanguage{JavaScript}{
  morekeywords=[1]{break, continue, delete, else, for, function, if, in,
    new, return, this, typeof, var, void, while, with},
  % Literals, primitive types, and reference types.
  morekeywords=[2]{false, null, true, boolean, number, undefined,
    Array, Boolean, Date, Math, Number, String, Object},
  % Built-ins.
  morekeywords=[3]{eval, parseInt, parseFloat, escape, unescape},
  sensitive,
  morecomment=[s]{/*}{*/},
  morecomment=[l]//,
  morecomment=[s]{/**}{*/}, % JavaDoc style comments
  morestring=[b]',
  morestring=[b]"
}[keywords, comments, strings]

\lstdefinelanguage[ECMAScript2015]{JavaScript}[]{JavaScript}{
  morekeywords=[1]{await, async, case, catch, class, const, default, do,
    enum, export, extends, finally, from, implements, import, instanceof,
    let, static, super, switch, throw, try},
  morestring=[b]` % Interpolation strings.
}

\lstalias[]{ES6}[ECMAScript2015]{JavaScript}

\lstdefinestyle{JSES6Base}{
  backgroundcolor=\color{white},
  basicstyle=\ttfamily,
  breakatwhitespace=false,
  breaklines=false,
  %captionpos=b,
  columns=fullflexible,
  commentstyle=\color{mediumgray}\upshape,
  emph={},
  emphstyle=\color{crimson},
  extendedchars=true,  % requires inputenc
  fontadjust=true,
  frame=single,
  identifierstyle=\color{black},
  keepspaces=true,
  keywordstyle=\color{mediumblue},
  keywordstyle={[2]\color{darkviolet}},
  keywordstyle={[3]\color{royalblue}},
  numbers=left,
  numbersep=5pt,
  numberstyle=\tiny\color{black},
  rulecolor=\color{black},
  showlines=true,
  showspaces=false,
  showstringspaces=false,
  showtabs=false,
  stringstyle=\color{forestgreen},
  tabsize=2,
  title=\lstname,
  upquote=true  % requires textcomp
}

\lstdefinestyle{JavaScript}{
  language=JavaScript,
  style=JSES6Base
}
\lstdefinestyle{ES6}{
  language=ES6,
  style=JSES6Base
}

\lstdefinelanguage{json}{
    basicstyle=\color{json_base}\normalfont\ttfamily,
    numbers=left,
    numberstyle=\scriptsize,
    stepnumber=1,
    numbersep=8pt,
    showstringspaces=false,
    breaklines=true,
    frame=single,
    backgroundcolor=\color{white},
    %captionpos=b,
    literate=
     *{0}{{{\color{numb}0}}}{1}
      {1}{{{\color{numb}1}}}{1}
      {2}{{{\color{numb}2}}}{1}
      {3}{{{\color{numb}3}}}{1}
      {4}{{{\color{numb}4}}}{1}
      {5}{{{\color{numb}5}}}{1}
      {6}{{{\color{numb}6}}}{1}
      {7}{{{\color{numb}7}}}{1}
      {8}{{{\color{numb}8}}}{1}
      {9}{{{\color{numb}9}}}{1}
      {:}{{{\color{punct}{:}}}}{1}
      {,}{{{\color{punct}{,}}}}{1}
      {\{}{{{\color{delim}{\{}}}}{1}
      {\}}{{{\color{delim}{\}}}}}{1}
      {[}{{{\color{delim}{[}}}}{1}
      {]}{{{\color{delim}{]}}}}{1}
}

\lstset{upquote=true}

\lstdefinestyle{mystyle}{
    backgroundcolor=\color{white},
    frame=single,
    commentstyle=\color{codegreen},
    keywordstyle=\color{codepurple},
    numberstyle=\numberstyle,
    stringstyle=\color{codepurple},
    basicstyle=\footnotesize\ttfamily,
    breakatwhitespace=false,
    breaklines=true,
    %captionpos=b,
    keepspaces=true,
    numbers=left,
    numbersep=10pt,
    showspaces=false,
    showstringspaces=false,
    showtabs=false,
}
\lstset{style=mystyle}

\newcommand\numberstyle[1]{%
    \footnotesize
    \color{codegray}%
    \ttfamily
    \ifnum#1<10 0\fi#1 |%
}
% CONFIGURAÇÕES DE PACOTES
% Configurações do pacote backref
% Usado sem a opção hyperpageref de backref
\renewcommand{\backrefpagesname}{Citado na(s) página(s):~}
% Texto padrão antes do número das páginas
\renewcommand{\backref}{}
%Define os textos da citação
\renewcommand*{\backrefalt}[4]{
	\ifcase #1 %
		Nenhuma citação no texto.%
	\or
		Citado na página #2.%
	\else
		Citado #1 vezes nas páginas #2.%
	\fi}

% ------------------------
% atalhos
\titulo{TÍTULO DO TRABALHO: SUBTÍTULO (se houver)}
\autor{Emanuel Moraes de Almeida}
\local{Brasil}
\data{25 de setembro de 2017}
\instituicao{Instituto Federal de Educação, Ciência e Tecnologia do Ceará}
\tipotrabalho{Trabalho de Conclusão de Curso (TCC)}
% O preambulo deve conter o tipo do trabalho, o objetivo, o nome da instituição e a área de concentração 
\preambulo{Modelo canônico de Relatório Técnico e/ou Científico em conformidade
com as normas ABNT apresentado à comunidade de usuários \LaTeX.}


% ------------------------
% Configurações de aparência do PDF final
% alterando o aspecto da cor azul
\definecolor{blue}{RGB}{41,5,195}
%informações do PDF
\makeatletter
\hypersetup{
    %pagebackref=true,
	pdftitle={\@title}, 
	pdfauthor={\@author},
    pdfsubject={\imprimirpreambulo},
	pdfcreator={LaTeX with abnTeX2},
	%pdfkeywords={abnt}{latex}{abntex}{abntex2}{relatório técnico}, 
	colorlinks=true,% false: boxed links; true: colored links
    linkcolor=blue,	% color of internal links
    citecolor=blue,	% color of links to bibliography
    filecolor=magenta, % color of file links
	urlcolor=blue,
	bookmarksdepth=4
}
\makeatother


% ------------------------
% Espaçamentos entre linhas e parágrafos 
% O tamanho do parágrafo é dado por:
\setlength{\parindent}{1.5cm}
% Controle do espaçamento entre um parágrafo e outro:
\setlength{\parskip}{0.2cm}  % tente também \onelineskip

% compila o indice
\makeindex
%\input{modeloCapitulos}

\begin{document}

%PARA UTILIZAR ARIAL
\fontfamily{phv}  %fonte Arial
\renewcommand{\rmdefault}{phv}      

% Retira espaço extra obsoleto entre as frases.
%\frenchspacing 
\thispagestyle{empty}
\vfill
\begin{center}

\begin{figure}[t]
\centering
\includegraphics[width=4cm]{figuras/ifce-ceara.png}%\\[-0.01in]
\end{figure}
\vspace{0.5 cm}
{\normalsize\bfseries INSTITUTO FEDERAL DE EDUCAÇÃO, CIÊNCIA E TECNOLOGIA DO CEARÁ} \\
{\normalsize\bfseries IFCE CAMPUS ARACATI} \\
{\normalsize\bfseries COORDENADORIA DE CIÊNCIA DA COMPUTAÇÃO}  \\ 
{\normalsize\bfseries BACHARELADO EM CIÊNCIA DA COMPUTAÇÃO}  \\ 

\vspace*{1in}
\begin{large} \bfseries \uppercase{Emanuel Moraes de Almeida} \end{large}\\[0.4in]

\vspace*{4cm}
\noindent \\
\large\bfseries{\uppercase{\textit{Alpha} \textit{Restful}: Um \textit{Framework} para Otimizações na Implementação de Funcionalidades Sobre Dados Normalizados no \textit{MongoDB}}} \\
\vfill
\normalsize\bfseries{ARACATI-CE\\\today}

\end{center}
\normalsize
\vfill
\begin{center}

{\imprimirautor\\}
\vspace{3cm}
{\textsc{\uppercase{Desenvolvimento e Aplicação de um Framework para Otimizações na Implementação de Funcionalidades Sobre Dados Normalizados no \textit{MongoDB}}}\\}
\vspace{5cm}
\hspace{.45\linewidth}
\begin{minipage}{.50\linewidth}
Trabalho de Conclusão de Curso (TCC) apresentado ao curso de Bacharelado em Ciência da Computação do Instituto Federal de Educação, Ciência e Tecnologia do Ceará - IFCE - Campus Aracati, como requisito parcial para obtenção do Título de Bacharel em Ciência da Computação.

\vspace{0.5 cm}

Orientador: Me. Diego Rocha Lima\\
Coorientador: \raggedright{Esp. Thiago Felippe de Lima}

\end{minipage}

\vspace{2cm}
\vfill
{\large Aracati-CE\\\today}

\end{center}

%\include{outrasFolhas/FichaCatalografica}
\begin{folhadeaprovacao}
\vfill
\begin{center}

{Emanuel Moraes de Almeida\\}
\vspace{1.5cm}
{\textsc{\uppercase{Desenvolvimento e Aplicação de um Framework para Otimizações na Implementação de Funcionalidades Sobre Dados Normalizados no \textit{MongoDB}}}\\}
\vspace{1.5cm}
\hspace{.45\linewidth}
\begin{minipage}{.50\linewidth}
Trabalho de Conclusão de Curso (TCC) apresentado ao curso de Bacharelado em Ciência da Computação do Instituto Federal de Educação, Ciência e Tecnologia do Ceará - IFCE - Campus Aracati, como requisito parcial para obtenção do Título de Bacharel em Ciência da Computação. 
\end{minipage}
\vspace{1.0 cm}

\end{center}

\noindent\\
{Aprovada em <data>}

\vspace{1.5 cm}
\begin{center}
{BANCA EXAMINADORA}
\assinatura{Prof. <Título Abreviado> <Nome completo> (Orientador (a)) \\ 
<instituição>}
\assinatura{Prof. <Título Abreviado> <Nome completo> (Orientador (a)) \\<instituição>}
\assinatura{Prof. <Título Abreviado> <Nome completo> (Orientador (a)) \\<instituição>}

\end{center}
\end{folhadeaprovacao}
\vfill
\begin{center}
{\textbf{DEDICATÓRIA}\\}
\end{center}

\noindent Aos meus pais.\\
\noindent Aos mestres.\\


\vfill
\begin{center}
{\textbf{AGRADECIMENTOS}}
\end{center}

Agradecimentos aqui.



\vfill
\begin{center}
{\textbf{RESUMO}\\}
\end{center}
\noindent

Com a crescente demanda para armazenar uma quantidade cada vez maior de dados, os modelos de bancos de dados tradicionais vêm demonstrando dificuldades no desenvolvimento de aplicações escaláveis, disponíveis e consistentes. Por isso, o modelo NoSQL tornou-se uma tecnologia emergente para suprir as deficiências do modelo relacional. O \textit{MongoDB}, que está entre os bancos de dados NoSQL mais populares, apesar de facilitar o armazenamento de dados desnormalizados, muitas vezes não possui recursos simples e intuitivos para a manipulação de dados normalizados. Neste contexto, este trabalho apresenta um \textit{framework}, denominado \textit{Alpha Restful}, que propõe facilitar o desenvolvimento de funcionalidades sobre dados normalizados utilizando o \textit{MongoDB}. Como resultado, através de exemplos de código fonte, demonstrou-se de que forma o \textit{Alpha Restful} se sobressai em relação às alternativas apresentadas, comprovando a relevância deste estudo.

\vspace{\onelineskip}
 \noindent
 \textbf{Palavras-chave}: \textit{MongoDB}; Normalização; \textit{Framework}; \textit{Alpha Restful}.

\vfill
\begin{center}
{\textbf{ABSTRACT}\\}
\end{center}

\noindent

Abstract here.
 
 \vspace{\onelineskip}
    
 \noindent
 \textbf{Keywords}: First. Second. Third.

\include{outrasFolhas/ListaIlustracoes}
\vfill
\begin{center}
{\textbf{LISTA DE TABELAS}}
\end{center}
\renewcommand\listtablename{}
\listoftables*
\vfill
\begin{center}
{\textbf{LISTA DE ABREVIATURAS E SIGLAS}}
\end{center}
\vspace{0.5cm}

\begin{siglas}
% \item[IFCE] Instituto Federal de Educação, Ciência e Tecnologia do Ceará
% \item[TCC] Trabalho de Conclusão de Curso
\item[SQL] \textit{Structured Query Language}
\item[NoSQL] \textit{Not Only SQL}
\item[CRUD] \textit{Create, Read, Update e Delete}
\item[IoT] \textit{Internet of Things}
\item[JSON] \textit{JavaScript Object Notation}
\item[BJSON] \textit{Binary JSON}
\end{siglas}

\include{outrasFolhas/Sumario}
\textual

%CAPITULOS
\chapter{INTRODUÇÃO}
\label{Introducao}

A grande demanda pela informatização de processos, que antes eram desenvolvidos completamente por humanos, tem sido bastante crescente. Isto vem estimulando a criação de novas ferramentas para facilitar o desenvolvimento de aplicações que automatizam tais processos. Para que tais aplicações possam se comunicar, torna-se necessário inseri-lo em uma rede local ou global de computadores. Dessa forma, o desenvolvimento \textit{Web} é usado para que a comunicação na rede do sistema desenvolvido possa acontecer.

%Uma aplicação pode ser desenvolvida em diversas plataformas. \hl{Uma das plataformas mais utilizadas atualmente é a web}, que permite a comunicação entre dispositivos localizados no mundo inteiro que se conectam a ela.

%%%%%
% A web pode realmente ser definida como uma plataforma?
%%%%%

%É comum uma aplicação web obedecer a um determinado padrão arquitetônico que irá definir regras para a organização dos seus pontos de acesso. Um dos padrões arquitetônicos mais utilizados é o REST.

\hl{Dentre as características mais comuns no desenvolvimento das mais diversas aplicações web}, estão a necessidade de armazenar, buscar, atualizar e remover dados, popularmente conhecido como CRUD (\textit{Create}, \textit{Read}, \textit{Update} e \textit{Delete}). Dependendo do propósito da aplicação desenvolvida, todas ou algumas dessas operações são utilizadas. \hl{Uma das abordagens mais antigas de armazenamento de dados é a utilização de arquivos de texto}, mas com o passar do tempo, \hl{gerenciar tais dados dessa maneira se tornou ineficiente}. Motivados por esta problemática, os Banco de Dados foram criados.

%%%%%
% Será que realmente deve ser afirmado que "Dentre as características mais comuns no desenvolvimento das mais diversas aplicações web"?

% Eu preciso provar que "Uma das abordagens mais antigas de armazenamento de dados é a utilização de arquivos de texto"?

% Eu preciso provar que "gerenciar tais dados dessa maneira se tornou ineficiente"?
%%%%%

% \hl{Um Banco de Dados é um sistema que possibilita o registro de dados seguindo uma determinada estrutura de armazenamento}

Assim como descrito por \cite{date2004introduccao}, um banco de dados é ``um sistema de armazenamento de dados baseado em computador; isto é, um sistema cujo objetivo global é registrar e manter informação''. \hl{Um SGBD (Sistema de Gerenciamento de Banco de Dados) é um software capaz de manipular um determinado banco de dados, disponibilizando para o usuário ferramentas de CRUD}. \hl{Um SGBD realiza todas as operações e tratamentos necessários, fornecendo um \textit{endpoint} de inserção de comandos para o gerenciamento mais simples, consistente e performático da manipulação de dados.}

%%%%%
% Buscar definição de banco de dados da literatura

% Buscar definição de SGBD da literatura

% Buscar na literatura as facilidades disponibilizadas por um SGBD
%%%%%

\hl{Vários tipos diferentes de Banco de Dados foram criados ao longo do tempo}, mas os Bancos de Dados \hl{atualmente mais utilizados pelas empresas seguem o modelo relacional}. \hl{Tal modelo define uma estrutura de dados normalizada baseado em tabelas que se relacionam entre si}. \hl{Tal modelo demonstrou-se consistente e performático, popularizando-se rapidamente}. \hl{A linguagem padrão utilizada por tais bancos é o SQL (Structured Query Languagem)}.

%%%%%
% Citar exemplos de tipos de bancos de dados

% Provar que o modelo mais utilizado atualmente é o modelo relacional

% Buscar definição de modelo relacional na literatura

% Fundamentar cientificamente que o modelo relacional "demonstrou-se consistente e performático, popularizando-se rapidamente"

% Quem definiu que o SQL é a linguagem padrão do modelo relacional? É necessário explicitar isso?
%%%%%

O modelo Relacional foi desenvolvido visando a imposição de alguns limites. Tais limites são definidos pelas regras de normalização \hl{e foram motivados para}:

%%%%%
% Justificar usando a literatura tais motivações
%%%%%

\begin{itemize}
	\item Otimizar a quantidade de dados armazenados;
	\item Otimizar a atualização de registros;
	\item Criar regras na própria estrutura de armazenamento, a fim de dificultar a inconsistência de dados.
\end{itemize}

\hl{Com o passar do tempo, os dispositivos começaram a aumentar sua capacidade de armazenamento drasticamente}. Tal avanço tecnológico estimulou a criação de aplicações que necessitam armazenar uma quantidade cada vez maior de dados. Mediante tal cenário, \hl{pôde-se observar} que os limites impostos pela normalização vêm demonstrando ser uma barreira tecnológica que dificulta a criação de aplicações altamente escaláveis, disponíveis e consistentes.

%%%%%
% Citar exemplos e estatísticas de como os dispositivos aumentaram sua capacidade de armazenamento

% Pôde-se observar como? Citar trabalhos relacionados que demonstre isso
%%%%%

Baseado nessa problemática, começaram a surgir novos modelos de armazenamento de dados, objetivando melhorar a performance de aplicações cujo o atendimento de suas exigências fosse muito caro, complexo ou inviável para o modelo de banco de dados Relacional.

\hl{Os bancos que estruturam seus dados usando abordagens não relacionais ou parcialmente relacionais são denominados de NoSQL (Not Only SQL)}. Vários tipos diferentes de banco de dados NoSQL foram surgindo, como por exemplo, os bancos baseados em células de \hl{tuplas, grafos, chave-valor e documento}.

%%%%%
% Buscar definição de NoSQL na literatura

% Referenciar trabalhos para esses tipos de NoSQL não abordados para demonstrar que eles existem
%%%%%

Atualmente, um dos banco de dados NoSQL \hl{mais utilizados é o \textit{MongoDB}}. Tal banco estrutura seus dados baseado em documentos. Esta abordagem quebra várias barreiras limitadas pelo modelo Relacional, permitindo que os dados sejam armazenados de maneira desnormalizada.

%%%%%
% Será que o MongoDB é um dos mais utilizados? Procurar referências na literatura para isso
%%%%%

A desnormalização permite uma maior flexibilização da estrutura de armazenamento, \hl{possibilitando a utilização de diversas técnicas específicas} para vários tipos de aplicações. Apesar dos benefícios da desnormalização, existem possíveis problemas que podem decorrer mediante seu uso.

%%%%%
% Que técnicas são permitidas pela flexibilização da estrutura de armazenamento? Onde na literatura isso é dito?
%%%%%

\hl{Os limites impostos pela normalização tentam garantir que um determinado valor somente precisará ser alterado em um único lugar}. Por causa desta característica, \hl{independente da complexidade do banco, a atualização de um valor, no geral, é bastante rápida}. Em contra partida, um banco desnormalizado não garante que um determinado valor estará em um único lugar, podendo exigir buscas possivelmente lentas para a localização de todos os locais onde o dado deve ser atualizado.

%%%%%
% Procurar na literatura que a normalização tende a "garantir que um determinado valor somente precisará ser alterado em um único lugar"

% Será mesmo que a atualização sempre tende a ser rápida no modelo relacional?
%%%%%

Por causa de tal problemática, aplicações que usam, por exemplo, o \textit{MongoDB}, podem sentir a necessidade de mesclar estratégias de normalização e desnormalização. Com o \textit{MongoDB}, os dados podem estar normalizados, parcialmente normalizados ou desnormalizados.

O \textit{MongoDB} foi projetado para que os dados possam estar desnormalizados e ele oferece \hl{um suporte muito bom para isto}, mas ele, atualmente, não oferece um suporte tão bom quanto em bancos relacionais para dados normalizados. Por causa disto, \hl{existem operações em dados normalizados que seriam mais simples de serem realizados usando a linguagem SQL, mas que se tornam mais difíceis de serem realizados no \textit{MongoDB}}. Dentro de um ambiente de desenvolvimento web com \textit{MongoDB}, tais problemáticas podem ser frequentemente encontradas.

%%%%%
% Justificar a afirmação do suporte ser muito bom para dados desnormalizados

% Justificar e demonstrar como e que operações são mais simples no SQL do que no MongoDB
%%%%%

O desenvolvimento de uma aplicação web com \textit{MongoDB} pode exigir um trabalho extra, em comparação com bancos relacionais, pois, as vezes, será necessário realizar uniões e buscas em dados normalizados ou parcialmente normalizados de forma não \hl{muito} intuitiva. Além disto, por causa da alta flexibilidade na estruturação dos dados, as diferentes formas como os dados podem ser armazenados exigem tratamentos \hl{muitos} distintos na hora de implementar funcionalidades comuns para o desenvolvimento web.

Dessa forma, se a estrutura dos dados mudar durante o desenvolvimento, as operações de busca e união precisarão ser alteradas. Ou seja, haverão mudanças em todas as funcionalidades que estiverem referenciando os dados reestruturados.

Além dos problemas decorrentes do uso do \textit{MongoDB}, uma aplicação web exige a implementação de um conjunto de funcionalidades que, apesar de serem comuns para este tipo de aplicação, podem exigir \hl{muito} trabalho para serem desenvolvidas.

Visando a resolução de tais problemas, este trabalho apresenta o desenvolvimento de um framework denominado \textit{Alpha Restful}, criado para desenvolver aplicações web com \textit{MongoDB} na linguagem \textit{ECMAScript} 6 (\textit{JavaScript}) e \textit{NodeJS} (no mínimo na versão 8).

\hl{Um framework é um conjunto de códigos fonte que oferece camadas de abstração para facilitar o desenvolvimento de um conjunto de funcionalidades disponibilizadas por ele}.

%%%%%
% Buscar na literatura a definição de framework
%%%%%

O \textit{Alpha Restful} abstrai a implementação de diversas funcionalidades que precisariam ser desenvolvidas manualmente. O desenvolvimento feito usando tal framework torna-se mais simples, pois ele obtém informações sobre a forma como os dados estão armazenados. Essas informações são utilizadas para que diversas funcionalidades sejam abstraídas em implementações mais simples e direcionadas ao que realmente se deseja fazer. 

%O \textit{Alpha Restful} pode ser dividido em 3 camadas de abstração, que serão explicadas posteriormente.

%A primeira camada é a de modelagem. Nela o desenvolvedor irá descrever a forma como os dados serão armazenados. Nesta camada será possível criar um padrão a ser seguido na hora de salvar os dados. Tal padrão irá definir quais dados estarão normalizados, desnormalizados ou parcialmente normalizados. Nesta camada, será possível informar comportamentos específicos para cada valor armazenado.

%Nesta etapa o framework irá gerar metadados (conjunto de dados que descrevem informações sobre a forma como os dados estão estruturados). Tais metadados serão utilizados pelas camadas seguintes a fim de automatizar a implementação de várias funcionalidades. Com os dados da modelagem (descrição feita pelo usuário) e com os metadados (dados gerados pelo \textit{Alpha Restful}), diversas funcionalidades serão implementadas pelo próprio framework, exigindo apenas que o usuário informe informações descritivas na hora de desenvolver tais funções. Por causa de tal abstração, pesquisas podem ser realizadas da mesma maneira, independente dos dados estarem normalizados, desnormalizados ou parcialmente normalizados.

%A segunda camada é a de CRUD automático. Nela o framework poderá criar (bastando apenas a habilitação desta opção) automaticamente todo o CRUD da aplicação. Operações de registro, remoção, atualização e busca serão criados automaticamente, sem que nenhuma codificação seja necessária.

%Nesta etapa, o framework irá implementar uma funcionalidade de busca padrão. Nela, serão utilizados os dados e metadados da modelagem para prever uma enorme quantidade de possibilidades de pesquisas possíveis. A customização desta camada é limitada em um conjunto enumerável de opções. Para aplicações simples, o \textit{Alpha Restful} exigirá apenas a definição da modelagem e o sistema estará pronto.

%Ainda nesta camada, alguns comportamentos padrão serão automaticamente executados, tendo como base as informações definidas na primeira camada. Um destes comportamentos é a relação de dependência entre os documentos. Isto permite que entidades sejam removidas em cascata, ou que sejam desreferenciadas (uma determinada chave estrangeira seja removida de uma entidade relacionada) ou ainda que sejam impedidas de serem removidas tendo como base a remoção de uma outra entidade relacionada.

%A segunda camada é muito poderosa e implementa diversas funcionalidades automaticamente tendo como base a modelagem definida pelo usuário. As customizações podem ser realizadas na modelagem dos dados, mas o número de customizações são finitas e enumeráveis, apesar de bastante poderosas e poderem ser usadas recursivamente.

%Caso a segunda camada não seja o suficiente, é disponibilizado uma terceira camada de abstração. Nela, um conjunto de funcionalidades bastante comuns em aplicações web REST podem ser definidas. Nesta camada, várias funcionalidades poderão ser feitas baseando-se em um padrão descritivo, ou seja, ao invés de o desenvolvedor precisar chamar as funções disponibilizadas pelo \textit{MongoDB}, o próprio framework se encarregará de realizar esta chamada. A única coisa que o desenvolvedor precisará fazer é disponibilizar um objeto que contém a descrição do que ele deseja que seja feito. Esta descrição é algo bem enxuto, tendo em vista que o \textit{Alpha Restful} já conhece a forma pela qual os dados estão estruturados e já possui diversos dados e metadados sobre comportamentos específicos de cada valor salvo definido na camada de modelagem. As pesquisas implementadas nesta camada, por causa de sua abstração, são realizados da mesma maneira, independente dos dados estarem normalizados, desnormalizados ou parcialmente normalizados.

\section{Objetivos}
\subsection{Objetivo Geral}
\subsection{Objetivos Específicos}

\section{Organização do Trabalho}
\chapter{FUNDAMENTAÇÃO TEÓRICA}
\label{FundamentacaoTeorica}

\section{Modelo de Banco de Dados Relacional}

O modelo Relacional surgiu por volta dos anos de 1970, com base no modelo proposto por Codd \cite{codd1970relational}. Através desse modelo, os dados são armazenados com um forte grau de independência, desacoplando a lógica, da representação dos dados \cite{davoudian2018survey}. Na representação Relacional, os dados são normalizados e as diversas relações (tabelas) podem referenciar os dados contidos em outras tabelas.

\hl{Cada linha de uma tabela (tupla) deve representar a abstração de um objeto do sistema}, sendo que \hl{cada célula da tupla é uma característica (atributo) do objeto representado}. Cada tupla \hl{deve possuir} pelo menos uma célula como identificador único (chave primária) que irá \hl{representar todos os dados contido na tupla} a qual ela está contida. A chave primária é um \hl{valor único} para a tabela, geralmente representado com um tipo numérico.

%%%%%
% Referenciar trabalhos científicos ou livros que apresentem as informações destacadas no parágrafo anterior
%%%%%

Caso seja necessário armazenar os dados contidos em uma tupla de outra relação, basta, em alguma célula da tupla, armazenar uma \hl{chave estrangeira} de outra tabela. Uma chave estrangeira é a chave primária de uma tupla de outra relação. Ao armazenar como chave estrangeira a chave primária de outra tabela, é realizado um link lógico, simbolizando que todos os dados representados por esta chave estrangeira também estão contidos dentro desta mesma tupla.

%%%%%
% Refereciar trabalhos científicos ou livros que apresentem as informações destracadas sobre chave estrangeira
%%%%%

Como exemplo, pode-se abstrair um sistema que precisa armazenar várias pessoas, na qual cada uma possui um nome e uma idade. \hl{Neste exemplo} pode-se representar os dados da seguinte forma:

\begin{table}[h]
    \centering
    \begin{tabular}{|c|c|c|}
        \hline
        \rowcolor[HTML]{FFAC71} 
        \multicolumn{3}{|c|}{\cellcolor[HTML]{FFAC71}Pessoa}                                                                                          \\ \hline
        \rowcolor[HTML]{9698ED} 
        {\color[HTML]{000000} \begin{tabular}[c]{@{}c@{}}Chave \\ Primária\end{tabular}} & {\color[HTML]{000000} Nome} & {\color[HTML]{000000} Idade} \\ \hline
        1                                                                                & Emanuel                     & 21                           \\ \hline
        2                                                                                & Eduardo                     & 40                           \\ \hline
    \end{tabular}
\end{table}

\hl{Nesse exemplo}, pode-se também desejar armazenar várias casas, na qual uma casa terá a sua rua e número. Neste caso é possível criar uma nova relação para a entidade (abstração de um tipo de objeto em nosso sistema) Casa:

\newpage

\begin{table}[h]
    \centering{
        \begin{tabular}{|c|c|c|}
            \hline
            \rowcolor[HTML]{67FD9A} 
            \multicolumn{3}{|c|}{\cellcolor[HTML]{67FD9A}Casa}                                                                                            \\ \hline
            \rowcolor[HTML]{9698ED} 
            {\color[HTML]{000000} \begin{tabular}[c]{@{}c@{}}Chave \\ Primária\end{tabular}} & {\color[HTML]{000000} Rua} & {\color[HTML]{000000} Número} \\ \hline
            10                                                                               & Rua Castelo Branco         & 1B                            \\ \hline
            20                                                                               & Rua Pompel                 & 1089                          \\ \hline
        \end{tabular}
    }
\end{table}

Continuando com o exemplo, naturalmente, no desenvolvimento de um sistema, as entidades se relacionam entre si. Neste caso, uma pessoa pode possuir várias casas, mas uma casa possui apenas uma pessoa como dono. Para representar essa relação, precisa-se definir quem será o dono da relação, ou seja, quem irá receber a chave primária da outra entidade através de uma chave estrangeira. Sempre deverá existir apenas  um dono da relação, ou seja, a chave estrangeira desta relação deverá estar em apenas uma tabela.
    
Como uma pessoa pode possuir várias casas, o dono da relação deve ser a casa, caso contrário a tupla de uma pessoa teria que ser duplicada (a menos que uma terceira tabela seja criada). Para representar essa situação pode-se modelar os dados da seguinte forma:
    
\begin{table}[h]
        \centering
        \begin{tabular}{|c|c|c|}
            \hline
            \rowcolor[HTML]{FFAC71} 
            \multicolumn{3}{|c|}{\cellcolor[HTML]{FFAC71}Pessoa}                                                                                          \\ \hline
            \rowcolor[HTML]{9698ED} 
            {\color[HTML]{000000} \begin{tabular}[c]{@{}c@{}}Chave \\ Primária\end{tabular}} & {\color[HTML]{000000} Nome} & {\color[HTML]{000000} Idade} \\ \hline
            1                                                                                & Emanuel                     & 21                           \\ \hline
            2                                                                                & Eduardo                     & 40                           \\ \hline
        \end{tabular}
    \end{table}
    
    \begin{table}[h]
        \centering
        \begin{tabular}{|c|c|c|c|}
            \hline
            \rowcolor[HTML]{67FD9A} 
            \multicolumn{4}{|c|}{\cellcolor[HTML]{67FD9A}Casa}                                                                                                                                                                        \\ \hline
            \rowcolor[HTML]{9698ED} 
            {\color[HTML]{000000} \begin{tabular}[c]{@{}c@{}}Chave \\ Primária\end{tabular}} & {\color[HTML]{000000} Rua} & {\color[HTML]{000000} Número} & \begin{tabular}[c]{@{}c@{}}Chave \\ Estrangeira \\ de Pessoa\end{tabular} \\ \hline
            10                                                                               & Rua Castelo Branco         & 1B                            & 1                                                                         \\ \hline
            20                                                                               & Rua Pompel                 & 1089                          & 1                                                                         \\ \hline
        \end{tabular}
\end{table}

Observando os dados exemplo apresentados acima, a pessoa Emanuel possui as duas casas armazenados no banco de dados, enquanto o Eduardo não possui nenhuma casa.

\subsection{Normalização}

O modelo Relacional possui algumas regras que o desenvolvedor \hl{deverá observar} ao modelar as estruturas das relações de seu banco. Dentre essas regras estão contidas as regras de normalização.

%%%%%
% As regras de nornalização realmente devem ser seguidas em todas circunstâncias?
%%%%%

A normalização é um conjunto de regras \hl{que visam, principalmente}, organizar os dados de forma a reduzir a redundância e otimizar a quantidade de dados armazenados.

%%%%%
% Justificar através de citações o parâgrafo anterior
%%%%%

Tal normalização pode ser dividida em seis formas normais. A seguir, serão explicadas as três primeiras formas normais, por serem as mais importantes e serem a de impacto mais significante para os propósitos deste trabalho.

\subsubsection{Primeira Forma Normal}
    
Como visto anteriormente, uma tupla representa um objeto da entidade representada pela relação a qual a tupla pertence. Tendo esse princípio como base, como é possível representar a situação de uma casa possuir vários donos e ao mesmo tempo uma pessoa possuir várias casas?
    
O princípio da primeira forma normal define que nunca deve-se duplicar tuplas, garantindo que duas ou mais tuplas nunca representem o mesmo objeto. Para manter este princípio e ainda realizar um relacionamento de muitos para muitos (uma pessoa tem muitas casas e uma casa tem muitas pessoas), torna-se necessário a criação de uma tabela auxiliar para representar o relacionamento de Pessoa com Casa:
    
\begin{table}[h]
        \centering
        \begin{tabular}{|c|c|c|}
            \hline
            \rowcolor[HTML]{FFAC71} 
            \multicolumn{3}{|c|}{\cellcolor[HTML]{FFAC71}Pessoa}                                                                                          \\ \hline
            \rowcolor[HTML]{9698ED} 
            {\color[HTML]{000000} \begin{tabular}[c]{@{}c@{}}Chave \\ Primária\end{tabular}} & {\color[HTML]{000000} Nome} & {\color[HTML]{000000} Idade} \\ \hline
            1                                                                                & Emanuel                     & 21                           \\ \hline
            2                                                                                & Eduardo                     & 40                           \\ \hline
        \end{tabular}
\end{table}
    
\begin{table}[h]
        \centering
        \begin{tabular}{|c|c|c|}
            \hline
            \rowcolor[HTML]{FCFF2F} 
            \multicolumn{3}{|c|}{\cellcolor[HTML]{FCFF2F}RelacionamentoPessoaCasa}                                                                                                                                                                                                                                   \\ \hline
            \rowcolor[HTML]{9698ED} 
            {\color[HTML]{000000} \begin{tabular}[c]{@{}c@{}}Chave \\ Primária do\\ Relacionamento\end{tabular}} & {\color[HTML]{000000} \begin{tabular}[c]{@{}c@{}}Chave \\ Estrangeira \\ de Pessoa\end{tabular}} & {\color[HTML]{000000} \begin{tabular}[c]{@{}c@{}}Chave \\ Estrangeira \\ de Casa\end{tabular}} \\ \hline
            100                                                                                                    & 1                                                                                                & 10                                                                                             \\ \hline
            200                                                                                                    & 1                                                                                                & 20                                                                                             \\ \hline
            300                                                                                                    & 2                                                                                                & 10                                                                                             \\ \hline
        \end{tabular}
\end{table}
    
\begin{table}[h]
        \centering
        \begin{tabular}{|c|c|c|}
            \hline
            \rowcolor[HTML]{67FD9A} 
            \multicolumn{3}{|c|}{\cellcolor[HTML]{67FD9A}Casa}                                                                                            \\ \hline
            \rowcolor[HTML]{9698ED} 
            {\color[HTML]{000000} \begin{tabular}[c]{@{}c@{}}Chave \\ Primária\end{tabular}} & {\color[HTML]{000000} Rua} & {\color[HTML]{000000} Número} \\ \hline
            10                                                                               & Rua Castelo Branco         & 1B                            \\ \hline
            20                                                                               & Rua Pompel                 & 1089                          \\ \hline
        \end{tabular}
\end{table}
    
No exemplo de modelagem acima, o Emanuel possui as duas casas e o Eduardo possui a casa de número 1B.
    
\subsubsection{Segunda Forma Normal}
    
A segunda forma normal especifica que todos os dados de uma tupla devem ser representados por toda a sua chave primária e jamais por apenas parte dela.
    
Como exemplo disso pode-se afirmar que uma casa jamais poderia ser armazenado dentro da mesma tupla de uma pessoa, mesmo em um relacionamento um para um (Uma Pessoa para uma Casa e uma Casa para uma Pessoa). Isto ocorre pois os dados de uma casa são independentes dos dados de uma pessoa e a chave primária de uma pessoa não poderia representar também os dados de uma casa.
    
Uma chave primária também pode ser composta por várias células, mas todos os dados da tupla devem depender de todas as células da chave primária e caso exista algum dado que dependa apenas de parte da chave primária, uma nova tabela deve ser criada. Uma tabela somente está na segunda forma normal se também estiver na primeira forma normal.
    
\subsubsection{Terceira Forma Normal}
    
Continuando o exemplo, pode-se imaginar que devem existir atributos no relacionamento entre pessoa e casa, indicando o valor mensal que a pessoa deve pagar como aluguel pela casa, a quantidade de meses a pagar o aluguel e o total a pagar até o fim do contrato. Para isso, poderíamos adicionar tais atributos na tabela de relacionamento de Pessoa Com Casa:
    
    \begin{table}[h]
        \centering
        \begin{tabular}{|c|c|c|c|c|c|}
            \hline
            \rowcolor[HTML]{FCFF2F} 
            \multicolumn{6}{|c|}{\cellcolor[HTML]{FCFF2F}RelacionamentoPessoaCasa}                                                                                                                                                                                                                                                                                                                                                                                                                      \\ \hline
            \rowcolor[HTML]{9698ED} 
            {\color[HTML]{000000} \begin{tabular}[c]{@{}c@{}}Chave \\ Primária do\\ Relacionamento\end{tabular}} & {\color[HTML]{000000} \begin{tabular}[c]{@{}c@{}}Chave \\ Estrangeira \\ de Pessoa\end{tabular}} & {\color[HTML]{000000} \begin{tabular}[c]{@{}c@{}}Chave \\ Estrangeira \\ de Casa\end{tabular}} & \begin{tabular}[c]{@{}c@{}}Valor\\ Mensal\end{tabular} & \begin{tabular}[c]{@{}c@{}}Quantidade\\ de Meses\end{tabular} & \begin{tabular}[c]{@{}c@{}}Total a\\ Pagar\end{tabular} \\ \hline
            100                                                                                                    & 1                                                                                                & 10                                                                                             & 200                                                    & 12                                                            & 2400                                                    \\ \hline
            200                                                                                                    & 1                                                                                                & 20                                                                                             & 400                                                    & 24                                                            & 9600                                                    \\ \hline
            300                                                                                                    & 2                                                                                                & 10                                                                                             & 300                                                    & 8                                                             & 2400                                                    \\ \hline
        \end{tabular}
    \end{table}
    
Neste exemplo, O Emanuel paga 200 reais pela Casa de número 1B com contrato válido por 12 meses e paga 400 reais pela casa de número 1089 com contrato válido por 24 meses. O Eduardo paga 300 reais pela casa de número 1B com contrato válido por 8 meses.
    
Para este caso, esta relação não está na terceira forma normal, pois o atributo \textit{Total a Pagar} é dependente do \textit{Valor Mensal} e da \textit{Quantidade de Meses}. O total a pagar pode ser obtido através de uma busca no banco de dados realizando uma simples multiplicação do \textit{Valor Mensal} e da \textit{Quantidade de Meses}. Para que esta relação fique de acordo com a Terceira Forma Normal é necessário remover a coluna \textit{Total a Pagar}:
    
\begin{table}[h]
    \centering
    \begin{tabular}{|c|c|c|c|c|}
        \hline
        \rowcolor[HTML]{FCFF2F} 
        \multicolumn{5}{|c|}{\cellcolor[HTML]{FCFF2F}RelacionamentoPessoaCasa}                                                                                                                                                                                                                                                                                                                                                            \\ \hline
        \rowcolor[HTML]{9698ED} 
        {\color[HTML]{000000} \begin{tabular}[c]{@{}c@{}}Chave \\ Primária do\\ Relacionamento\end{tabular}} & {\color[HTML]{000000} \begin{tabular}[c]{@{}c@{}}Chave \\ Estrangeira \\ de Pessoa\end{tabular}} & {\color[HTML]{000000} \begin{tabular}[c]{@{}c@{}}Chave \\ Estrangeira \\ de Casa\end{tabular}} & \begin{tabular}[c]{@{}c@{}}Valor\\ Mensal\end{tabular} & \begin{tabular}[c]{@{}c@{}}Quantidade\\ de Meses\end{tabular} \\ \hline
        1                                                                                                    & 1                                                                                                & 10                                                                                             & 200                                                    & 12                                                            \\ \hline
        2                                                                                                    & 1                                                                                                & 20                                                                                             & 400                                                    & 24                                                            \\ \hline
        3                                                                                                    & 2                                                                                                & 10                                                                                             & 300                                                    & 8                                                             \\ \hline
    \end{tabular}
\end{table}
    
A terceira forma normal especifica que não deve existir uma célula que dependa de outras células. Cada valor deverá ser o mais independente possível dos outros valores. Para que uma relação esteja na terceira forma normal, também torna-se necessário que esteja na segunda forma normal.

\subsection{Junção de Tabelas}
    
Existem situações onde as tabelas precisam ser unidas em uma única tabela desnormalizada, para a realização de buscas mais complexas, ou para simplesmente agrupar os dados da maneira como eles devem ser consumidos nas aplicações que os irão utilizar.
    
Como exemplo disto, pode-se unir as três tabelas exemplo apresentadas anteriormente, apenas para a pessoa de nome \textit{Emanuel}. A seguir encontra-se uma codificação SQL para a união de tabelas e logo em seguida a tabela resultante de tal codificação.

\begin{lstlisting}[language=SQL, caption={União de Tabelas com SQL}]
    SELECT * 
    FROM Pessoa p JOIN
        RelacionamentoPessoaCasa rpc ON p.id = rpc.pessoa_id JOIN
        Casa c ON rpc.casa_id = c.id
    WHERE p.nome = "Emanuel"
\end{lstlisting}
    
    \begin{table}[h]
        \centering
        \resizebox{\textwidth}{!}{%
            \begin{tabular}{|c|c|c|c|c|c|c|c|c|c|c|}
                \hline
                \rowcolor[HTML]{38FFF8} 
                \multicolumn{11}{|c|}{\cellcolor[HTML]{38FFF8}Junção das Três Tabelas}                                                                                                                                                                                                                                                                                                                                                                                                                                                                                                                                                        \\ \hline
                \rowcolor[HTML]{9698ED} 
                \begin{tabular}[c]{@{}c@{}}Chave\\ Primária\\ de Pessoa\end{tabular} & Nome    & Idade & {\color[HTML]{000000} \begin{tabular}[c]{@{}c@{}}Chave \\ Primária do\\ Relacionamento\end{tabular}} & {\color[HTML]{000000} \begin{tabular}[c]{@{}c@{}}Chave \\ Estrangeira \\ de Pessoa\end{tabular}} & {\color[HTML]{000000} \begin{tabular}[c]{@{}c@{}}Chave \\ Estrangeira \\ de Casa\end{tabular}} & \begin{tabular}[c]{@{}c@{}}Valor\\ Mensal\end{tabular} & \begin{tabular}[c]{@{}c@{}}Quantidade\\ de Meses\end{tabular} & \begin{tabular}[c]{@{}c@{}}Chave\\ Primária\\ de Casa\end{tabular} & Rua                & Número \\ \hline
                1                                                                    & Emanuel & 21    & 100                                                                                                    & 1                                                                                                & 10                                                                                             & 200                                                    & 12                                                            & 10                                                                 & Rua Castelo Branco & 1B     \\ \hline
                1                                                                    & Emanuel & 21    & 200                                                                                                    & 1                                                                                                & 20                                                                                             & 400                                                    & 24                                                             & 20                                                                 & Rua Pompel & 1089     \\ \hline
            \end{tabular}%
        }
    \end{table}
    
Neste caso, algumas colunas do relacionamento de Pessoa com Casa são desnecessárias e no comando de junção das tabelas pode-se omití-las. Neste caso a tabela de junção seria algo como:
    
    \begin{table}[h]
        \centering
        \resizebox{\textwidth}{!}{%
            \begin{tabular}{|c|c|c|c|c|c|c|c|}
                \hline
                \rowcolor[HTML]{38FFF8} 
                \multicolumn{8}{|c|}{\cellcolor[HTML]{38FFF8}Junção das Três Tabelas}                                                                                                                                                                                                                                              \\ \hline
                \rowcolor[HTML]{9698ED} 
                \begin{tabular}[c]{@{}c@{}}Chave\\ Primária\\ de Pessoa\end{tabular} & Nome    & Idade & \begin{tabular}[c]{@{}c@{}}Valor\\ Mensal\end{tabular} & \begin{tabular}[c]{@{}c@{}}Quantidade\\ de Meses\end{tabular} & \begin{tabular}[c]{@{}c@{}}Chave\\ Primária\\ de Casa\end{tabular} & Rua                & Número \\ \hline
                1                                                                    & Emanuel & 21    & 200                                                    & 12                                                            & 10                                                                 & Rua Castelo Branco & 1B     \\ \hline
                1                                                                    & Emanuel & 21    & 400                                                    & 24                                                             & 20                                                                 & Rua Pompel & 1089     \\ \hline
            \end{tabular}%
        }
    \end{table}

\hl{A ideia dos dados estarem separados é de tentar garantir} que os dados fiquem consistentes na hora de atualizar os dados existentes, além de tentar garantir uma velocidade maior na hora de atualizar valores, mas em buscas no banco de dados, as junções de tabelas são, muitas vezes, inevitáveis.

%%%%%
% Referenciar trabalhos que demonstrem que a separação dos dados tem o proposito apresentado no parágrafo anterior
%%%%%
    
Em um sistema complexo, com uma grande quantidade de dados e relações, cada tabela poderá conter milhares ou até milhões de linhas, envolvendo a junção de centenas de tabelas, podendo deixar essas operações de junção lentas demais para os requisitos da aplicação. \hl{Existem algumas técnicas e estratégias} para deixar essas consultas mais rápidas, mas mesmo com essas estratégias, haverá aplicações na qual a exigência de alta performance, escalabilidade, disponibilidade e consistência tornará inviável ou altamente caro escalar uma aplicação utilizando o modelo Relacional.

%%%%%
% Que técnicas são essas faladas no parágrafo anterior? Elas realmente existem? Buscar referencias binliográficas para fazer tal afirmação
%%%%%

\subsection{Vantagens do Modelo Relacional}

%%%%%
% Justificar tais vantagens por meio de referências bibliográficas
%%%%%
    
\begin{itemize}
    \item O modelo Relacional já está bem solidificado no mercado, possuindo várias
         ferramentas e frameworks para facilitar o desenvolvimento dos mais diversos tipos
         de aplicações;
        
    \item Pelo fato do modelo Relacional forçar a separação e normalização dos dados,
         torna-se mais difícil que erros humanos ou de desenvolvimento venham a fazer os
         dados perderem sua consistência;
        
%    \item A normalização dos dados proporciona que os dados possam ser
%         inseridos/atualizados sem que o usuário/desenvolvedor tenha a responsabilidade de 
%         fazer diversas verificações físicas/lógicas na própria estrutura de armazenamento de
%         dados;
        
%    \item Uma modelagem de dados mal planejada possui menos chances de causar grandes
%         danos a consistência dos dados por causa da normalização;
        
    \item A normalização, por causa de seu princípio de não repetir dados em várias tabelas,
         garante que os dados sejam armazenados utilizando pouco espaço de
         armazenamento;

    \item Por causa da normalização, a atualização de registros são, geralmente, bastante 
         rápidas.
\end{itemize}
    
\subsection{Desvantagens do Modelo Relacional}

%%%%%
% Justificar tais desvantagens por meio de referências bibliográficas
%%%%%
    
\begin{itemize}
    \item O modelo de dados relacionais possui recursos de modelagem muito limitados;
        
    \item O mapeamento de objetos com operação de junção, muitas vezes são caras;
        
    \item As demandas recentes de armazenamento e consulta de uma grande quantidade 
         de dados revelaram várias deficiências do relacionamento relacional tradicional;
        
    \item Diversos tipos de aplicações com altos requisitos de escalabilidade, disponibilidade e
          consistência podem se tornam caras ou inviáveis;
        
    \item As aplicações precisam se adaptar ao modelo Relacional. Mesmo que uma aplicação
         possa ter uma parte dos dados não normalizados sem prejudicar a consistência por
         causa de alguma regra de negócio, o modelo relacional exige a normalização para
         um bom funcionamento do SGBD.
\end{itemize}

\section{Modelo de Banco de Dados NoSQL com \textit{MongoDB}}
    
Desde o início dos anos 2000, os avanços na tecnologia web, resultaram na explosão repentina de dados estruturados, semi-estruturados e não estruturados por aplicativos de escopo global. Tais aplicações geralmente exigem uma escalabilidade horizontal, adaptar-se às enormes quantidades de dados e à taxa crescente de processamento de consultas \cite{davoudian2018survey}.
    
A alta disponibilidade, a baixa tolerância a falhas para responder aos clientes, confiabilidade de transações, o suporte a dados altamente consistentes e a manutenção de \textit{schemas} de banco de dados com baixo custo de evolução do sistema são requisitos que se tornam muitos difíceis ou inatingíveis nos sistemas tradicionais de banco de dados Relacional \cite{davoudian2018survey}.
    
Além desse motivo, a ampliação de sistemas exige uma movimentação de servidores autônomos com hardware aprimorados, sendo um processo caro e causa uma indisponibilidade significativa a cada movimentação. Sistemas baseados em modelos Relacionais exigem uma complexidade e sobrecarga maiores para se juntar dados distribuídos normalizados \cite{davoudian2018survey}.
    
\hl{De acordo com publicações recentes}, os requisitos acima podem ser resolvidos compensando ou sacrificando outros requisitos não tão necessários para a aplicação.

%%%%%
% Que publicações recentes?
%%%%%
    
Os modelos de banco de dados NoSQL tornaram-se uma tendência emergente de armazenamento de dados não relacional, que visam satisfazer os requisitos de alta disponibilidade e escalabilidade de aplicações de âmbito global \cite{davoudian2018survey}.
    
Um modelo de Banco de Dados NoSQL é caracterizado pela utilização de um modelo não relacional ou parcialmente relacional para o armazenamento de dados. Vários modelos NoSQL foram criados visando esses princípios acima descritos. Para os propósitos deste trabalho, esta seção utiliza como base a descrição e utilização de modelos de banco de dados Baseado em Documento, sendo exemplificado pela descrição e uso do banco de dados \textit{MongoDB}.

Enquanto que no modelo Relacional o armazenamento ocorre em uma tabela, no MongoDB o armazenamento ocorre em uma coleção de documentos, sendo cada documento a representação de um objeto do sistema. Cada banco de dados NoSQL baseado em Documento terá a sua própria estrutura de documento. No caso específico do \textit{MongoDB}, os documentos são escritos de maneira estruturada, seguindo uma variação do formato JSON (\textit{JavaScript Object Notation}). Tal variação possui o nome de BSON (\textit{Binary} JSON).

\subsection{O Formato JSON}
    
O JSON é um formato de representação de dados \hl{chave-valor}. A chave é o nome do atributo (caracterísica) pertencente ao objeto na qual o JSON representa. Uma chave é utilizada para representar o valor (valor da característica do objeto) a ela associada.

%%%%%
% A representação de dados feita pelo JSON pode realmente ser chamada de "chave-valor"? Essa pergunta é feita pois existe uma outra representação de dados que possui esse nome
%%%%%

O valor associado a uma chave pode ser do tipo textual (entre aspas), tipo numérico (sem aspas), pode ser um outro JSON (definido entre chaves) e pode ser uma lista de qualquer um dos tipos definidos anteriormente (entre colchetes separados por vírgula).
    
A seguir é apresentado um exemplo de representação de uma lista de pessoas com suas respectivas casas:

\newpage

\begin{lstlisting}[language=json, caption={JSON Representando Uma Lista de Pessoas Com Suas Casas}]
[{
    "nome": "Emanuel",
    "idade": 21,
    "casas": [
        {
            "rua": "Rua Castelo Branco",
            "numero": "1B",
            "valorMensal": 200,
            "quantidadeMeses": 12
        },
        {
            "rua": "Rua Pompel",
            "numero": 1089,
            "valorMensal": 400,
            "quantidadeMeses": 24
        }
    ]
},
{
    "nome": "Eduardo",
    "idade": 40,
    "casas": {
        "rua": "Rua Pompel",
        "numero": "1089",
        "valorMensal": 300,
        "quantidadeMeses": 8
    }
}]
\end{lstlisting}

Observa-se que a representação de um objeto é dada entre chaves (\textit{\{\}}) e, dentro dessas chaves, são definidos os nomes dos atributos do objeto e, após os dois pontos (\textit{:}), é definido o valor para aquele atributo. Quando se deseja que o valor seja uma lista de elementos, envolve-se todos os elemento da lista entre colchetes e cada elemento é separado por vírgula (\textit{,}).
    
No exemplo apresentado acima, está sendo representado uma lista de objetos. Para a representação desta lista, as pessoas (JSON) são envolvidas entre colchetes. Caso seja desejado representar apenas uma pessoa ao invés de uma lista, bastaria remover os colchetes mais externos e deixar apenas um JSON na raiz do documento. Observa-se que a pessoa com o nome \textit{Eduardo} possui apenas uma casa, e por este motivo o valor do atributo \textit{casa} pode ser representado como um JSON, mas também seria possível representar como uma lista de apenas uma casa, bastando para isso apenas envolver as chaves (\textit{\{\}}) em colchetes (\textit{[]}), assim como ocorre para a pessoa \textit{Emanuel}.

\subsection{Coleções de Documentos no MongoDB}
    
O formato de documento utilizado pelo MongoDB é uma variação do JSON: o BSON (\textit{Binary JSON}). O BSON é quase idêntico ao JSON, sendo qua a diferença \hl{é apenas} a adição de novos tipos, como um tipo para a representação de datas e um tipo para a representação de um identificador para o documento (o \textit{ObjectId}).

%%%%%
% Será que não existe outra diferença entre o JSON e o BSON não apresentada no parágrafo anterior?
%%%%%

Enquanto que no modelo Relacional uma tabela representa uma coleção de objetos no sistema, no \textit{MongoDB}, esta representação é feita por uma coleção de documentos. Da mesma forma, enquanto uma tupla de uma tabela representa, no modelo Relacional, um objeto, no \textit{MongoDB}, esta representação é feita por um documento. Cada documento possui, em seu interior, um BSON (não uma lista, mas apenas um único BSON).
    
Seguindo essa abordagem, para, por exemplo, armazenarmos várias pessoas em \hl{nosso} banco, precisa ser criado uma coleção que irá armazenar vários documentos, sendo cada documento uma representação BSON de uma pessoa diferente. Cada documento terá um identificador único para a coleção (\textit{\_id}) que irá  representar todo o documento. Através deste identificador, pode-se fazer relações entre documentos. Tal identificador pode ser de qualquer tipo, porém o tipo \hl{recomentado} é o \textit{ObjectId}. A seguir é listado algumas vantagens do uso do \textit{ObjectId}:

%%%%%
% Utilização de pronome em primeira pessoa no parágrafo anterior

% O uso do ObjectId é recomendado por quem? Existem situações onde esse tipo não é recomendado?
%%%%%

\begin{itemize}
    \item O \textit{ObjectId} é formado por 12 bytes, sendo os quatro primeiros refletidos no \textit{timestamp} de quando o documento foi criado e possui uma \hl{probabilidade muito alta} de ser único. Por causa disto, é possível obter a data de criação de um documento pelo identificador
    
    \item Odenar pelo \textit{\_id} com o tipo \textit{ObjectId} é equivalente a ordenar os documentos pela data de criação (para os documentos criados no mesmo segundo não é garantido nenhuma ordem)
    
    \item O campo \textit{\_id} é automaticamente criado caso não seja informado
\end{itemize}

%%%%%
% Qual a probabilidade do ObjectId ser único? Usar o termo "muito" pode ser inadequado
%%%%%

A seguir será exemplificado um BSON que poderia ser armazenado em um documento do \textit{MongoDB}. Para ilustrar a flexibilidade do valor do \textit{\_id}, o identificador das entidades serão armazenados por meio de um valor numérico, porém os identificadores dos BSON ``internos'' serão definidos como um \textit{ObjectId}. Esta escolha foi tomada apenas para facilitar a visualização dos exemplos, mas em um sistema, o \textit{\_id} poderia assumir qualquer tipo de dado.

\begin{lstlisting}[language=json, caption={Estrutura de Dados da Pessoa \textit{Emanuel}}]
{
    "_id": 1,
    "nome": "Emanuel",
    "idade": 21,
    "casas": [
        {
            _id: ObjectId("5e66c969d5d7cc24c0850854"),
            "rua": "Rua Castelo Branco",
            "numero": "1B",
            "valorMensal": 200,
            "quantidadeMeses": 12
        },
        {
            _id: ObjectId("5e66c9740efbf528b05a7537"),
            "rua": "Rua Pompel",
            "numero": 1089,
            "valorMensal": 400,
            "quantidadeMeses": 24
        }
    ]
}
\end{lstlisting}

O identificador (\textit{\_id}) deve existir pelo menos no BSON principal (neste caso, o BSON que representa uma pessoa). O BSON que representa uma casa não necessariamente precisa de um identificador, mas, para este exemplo, optou-se por gerar um. O identificador do BSON de uma casa não é o identificador de outro documento, pois neste exemplo os dados estão desnormalizados e, por enquanto, para o nosso exemplo, não existe outra coleção além da coleção de Pessoa.

\subsection{Desnormalização}

A desnormalização é o armazenamento de dados sem a preocupação de seguir as formas normais. No exemplo apresentado, a coleção de Casa não existe e as casas estão armazenadas dentro do mesmo BSON na qual uma pessoa é armazenada. Tal estrutura desobedece a terceira forma normal, pois o identificador do documento (\textit{\_id}) identifica uma pessoa, não podendo, portanto, identificar os dados das casas que estão presente no próprio documento.
    
Uma das principais vantagens da desnormalização é que não existe a necessidade de realizar junção de vários documentos, podendo ter um ganho substancial de performance. No exemplo apresentado do modelo Relacional, foi necessário realizar buscas e uniões de tabelas para obter uma tabela contendo um balanço da lista total de casas da pessoa com nome \textit{Emanuel}. No MongoDB, caso desejemos realizar a mesma pesquisa para a pessoa \textit{Emanuel}, obtendo a lista de casas que ela possui, não haveria a necessidade de juntar quaisquer documentos, bastando apenas buscar o documento da pessoa a qual se deseja buscar e retornar o documento tal qual como foi armazenado.
    
No modelo relacional, como foi abordado anteriormente, uma tabela pode ter milhares ou milhões de linhas e pode ser necessário unir centenas de tabelas. Com a desnormalização, nenhuma junção torna-se necessária para este caso.

\subsection{Limites da Desnormalização}
    
Apesar dos benefícios da desnormalização, ela nem sempre é aconselhável ou viável. Através da desnormalização, os dados começam a ser duplicados, e dependendo das regras da aplicação, essa duplicação pode ser problemática. Como exemplo disso, pode-se imaginar uma situação onde várias pessoas possuem uma mesma casa. Neste exemplo, os dados da mesma casa precisariam ser duplicados em diversas pessoas. Se uma casa nunca precisar ser removida ou atualizada, essa duplicação possivelmente não irá gerar nenhum problema, mas caso uma casa precise ser removida ou ter, por exemplo o seu número atualizado, seria necessário atualizar a casa em todas as pessoas que a possuem. Essa operação poderia ser extremamente lenta e dependendo da quantidade de dados armazenados, isto seria uma operação inviável. A desnormalização deve ocorrer em situações onde as regras de negócio da aplicação garantem uma segurança na realização de tal procedimento.
    
Quando a desnormalização completa não for uma opção, torna-se necessário a criação de uma nova coleção para normalizar esses dados. Para exemplificar esta normalização, uma nova coleção deve ser criada para armazenar todas as casas, e no atributo \textit{casas} no documento de pessoa, apenas será armazenado o identificador da casa a qual ela está relacionada, junto com os atributos de relacionamento \textit{valorMensal} e \textit{quantidadeMeses}.

A seguir é apresentado exemplos de como os dados seriam estruturados mediante uma normalização:
    
\begin{lstlisting}[language=json, caption={Estrutura de Dados Normalizados de uma Casa}]
{
    "_id": 20,
    "rua": "Rua Pompel",
    "numero": "1089"
}
\end{lstlisting}
    
\begin{lstlisting}[language=json, caption={Estrutura de Dados Normalizados da Pessoa \textit{Eduardo}}]
{
    "_id": 2,
    "nome": "Eduardo",
    "idade": 40,
    "casas": {
        "_id": ObjectId("5e66cad68a1e581e3cf47dfc"),
        "id": 20,
        "valorMensal": 300,
        "quantidadeMeses": 8
    }
}
\end{lstlisting}

Observa-se que no atributo de \textit{casas} existem dois identificadores. O identificador \textit{\_id} identifica o BSON do relacionamento de pessoa com casa. O identificador \textit{id} é um atributo que poderia ter recebido outro nome. Como neste caso é necessário armazenar o identificador do documento de casa, escolhe-se um nome de atributo que será padronizado na aplicação para referenciar a casa a qual esta pessoa se relaciona. Neste exemplo, escolheu-se padronizar o atributo \textit{id} como aquele que receberá o identificador do documento da casa que se relaciona com esta pessoa. Desta forma, o identificador da casa a qual esta pessoa pertence (20) será armazenado no atributo \textit{id}. 

É possível padronizar para que o \textit{\_id} represente, ao mesmo tempo, o identificador do BSON (identificador do relacionamento) e o identificador da entidade relacionada. Nesse caso, o atributo \textit{id} não seria necessário. Para as exemplificações desse trabalho, essa padronização não será feita, pois ter esses dois identificadores pode gerar algumas vantagens em consultas. Uma possível vantagem seria a de uma entidade poder se relacionar mais de uma vez com o mesmo documento, de mesmo identificador (\textit{id}), porém o sistema poderia diferenciar os dois relacionamentos por meio do identificador da relação (\textit{\_id}). Além dessa vantagem, pode ser que alguma biblioteca gere o \textit{\_id} automaticamente e utilize esse valor gerado de alguma maneira (a biblioteca \textit{Mongoose}, que será explicada mais a frente, gera esse identificador automaticamente, caso não seja atribuído). Dessa forma, dependendo da situação, pode ser que seja uma boa ideia manter esses dois identificadores.
    
Nesse exemplo de normalização, os valores de uma casa (\textit{rua} e \textit{numero}) não são armazenados no documento de pessoa. Desta forma, caso uma casa tenha seu número atualizado, o número será atualizado no documento da casa e não precisará ser atualizado em todos os documentos das pessoas que se relacionam com esta casa.

\subsection{Desnormalização Parcial}
    
No exemplo apresentado na sessão anterior, uma das possíveis regras que poderia ser exigida é de que a rua de uma casa nunca possa ser atualizada, porém o número possa ser atualizado. Neste caso, pode-se realizar uma desnormalização parcial, normalizando apenas o atributo \textit{numero} e duplicando nas várias pessoas o valor do atributo de \textit{rua}. Nesta situação, não haveria necessidade de unir documentos para buscas que se interessem apenas no valor da rua das casas de uma determinada pessoa. Dependendo da naturesa da aplicação, pode haver situações onde a desnormalização pode ser utilizada, mesmo com a atualização do mesmo valor em vários documentos, caso as regras de negócio da aplicação garantam que os dados não serão duplicados o suficiente para trazer uma performance inaceitável.

\subsection{Junção de Documentos}
    
Quando se realiza uma normalização, poderá haver situações onde os documentos precisem ser unidos. No exemplo de normalização apresentado anteriormente, caso seja necessário unir os documentos da coleção de pessoa com os documentos da coleção de casa, seria obtido algo como:

\newpage

\begin{lstlisting}[language=json, caption={Junção de Documentos Normalizados}]
[{
    "_id": 1,
    "nome": "Emanuel",
    "idade": 21,
    "casas": [
        {
            "_id": ObjectId("5e66cb3a7216361ff05b3b8f"),
            "id": 10,
            "rua": "Rua Castelo Branco",
            "numero": "1B",
            "valorMensal": 200,
            "quantidadeMeses": 12
        },
        {
            "_id": ObjectId("5e66cb43534f8a1944cdb028"),
            "id": 20,
            "rua": "Rua Pompel",
            "numero": 1089,
            "valorMensal": 400,
            "quantidadeMeses": 24
        }
    ]
},
{
    "_id": 2,
    "nome": "Eduardo",
    "idade": 40,
    "casas": [{
        "_id": ObjectId("5e66cb581766a2056c48145f"),
        "id": 20,
        "rua": "Rua Pompel",
        "numero": "1089",
        "valorMensal": 300,
        "quantidadeMeses": 8
    }]
}]
\end{lstlisting}

A junção de documentos no MongoDB \hl{tende a ser mais rápido que as junções de tabelas no modelo Relacional}, pois as buscas por identificadores de documento no MongoDB são otimizadas, além de os atributos de relacionamento em relacionamentos muito para muitos não necessitar de um terceiro documento auxiliar, diminuindo a quantidade de documentos a serem unidos.

%%%%%
% Citações para comprovar que a junção de documentos tende a ser amis rápido que a junção de tabelas
%%%%%

\subsection{Vantagens do Banco de Dados \textit{MongoDB}}

%%%%%
% Citações para comprovar as vantagens
%%%%%
    
\begin{itemize}
    \item Os modelos de armazenamento de dados são muito flexíveis, sendo que cada documento pode ser estruturado de forma diferente uns dos outros
    
    \item Pela flexibilidade da modelagem, o armazenamento pode ser escalonado, podendo alcançar alta disponibilidade e baixo custo
    
    \item Por causa da flexibilidade da modelagem, os dados podem ser estruturados adaptando-se à aplicação, ao invés de a aplicação ter que se adaptar a modelagem
    
    \item Possibilitando a adição de dados relevantes em um mesmo documento e a duplicação dos dados em várias partes, a junção de dados possui um melhor desempenho, alcançando uma melhor velocidade e consulta
    
    \item Há um uso inteligente de índices distribuídos, como hashing e caching para acesso a dados e armazenamento
    
    \item Os dados podem ser facilmente replicados e particionados horizontalmente em servidores locais e remotos
    
    \item Fornece uma estrutura de dados muito similar e familiar com o tratamento de dados de aplicações web
\end{itemize}
    
\subsection{Desvantagens do Banco de dados \textit{MongoDB}}

%%%%%
% Citações para comprovar as vantagens
%%%%%
    
\begin{itemize}
    \item Não existem tantas ferramentas e frameworks para a utilização de banco de dados NoSQL como existem para a utilização de banco de dados Relacionais
    
%    \item Os dados podem precisar ser interpretados em nível de aplicação
    
    \item Pode levar a atualizações dispendiosas em dados duplicados
    
%    \item Exige uma responsabilidade maior por parte do desenvolvedor de criar uma modelagem de dados que não inviabilize a utilização de algumas operações básicas de manipulação dos dados
    
    \item Exige da parte da aplicação mais verificações na hora de alterar e
          remover dados
    
    \item Não trata automaticamente os relacionamentos manuais entre documentos descritos anteriormente, podendo deixar passar (caso o desenvolvedor não se atente) identificadores que apontam para documentos que já não existe mais
    
%    \item Uma modelagem de dados mal planejada pode causar danos catastróficos, possivelmente irreversíveis a consistência dos dados
    
    \item As buscas que precisam ser realizadas em documentos separados e relacionados podem ser mais complexas do que as consultas realizadas no modelo Relacional
\end{itemize}

%Exemplo de referência bibliográfica \cite{abntex2-wiki-como-customizar}.

%Exemplo de uso de figura no Latex (Figura~\ref{fig:mafalda}).
%\begin{figure}[th]
%    \centering{
%    \caption{Legenda da Figura no topo.}
%    \includegraphics[width=0.75\textwidth]{figuras/mafalda}
%    \begin{flushleft}
%    \flushleft{Fonte: 
%    Elaborado pelo autor.}
%    \end{flushleft}
%    \label{fig:mafalda}
%}
%\end{figure}


%Exemplo de uso de tabela no Latex (Tabela~\ref{tab:tabelaModelo}). Ver página %72 do Manual de Normalização de Trabalhos Acadêmicos do IFCE.
%\begin{table}[th]
%    \centering
%    \caption{Legenda da Tabela no Topo.}
%    \label{tab:tabelaModelo}
%    \begin{tabular}{llll}
%    & Nota mínima & Nota máxima & Nota média\\\hline
%    Ciências Humanas (CH)     & 324,8       & 862,1       & 546,5\\
%    Ciências da Natureza (CN) & 330,6       & 876,4       & 482,2\\
%    Linguagens e Códigos (LC) & 306,2       & 814,2       & 507,9\\
%    Matemática (MT)           & 318,5       & 973,6       & 473,5\\ \hline  
%    \end{tabular}
%    \begin{flushleft}
%    \flushleft{Fonte: 
%    Instituto Brasileiro de Geografia e Estatística - IBGE.}
%    \end{flushleft}
%\end{table}

\chapter{TRABALHOS RELACIONADOS}
\label{TrabalhosRelacionados}

O \textit{MongoDB} é um dos bancos de dados NoSQL mais populares e utilizados. Isso pode ser demonstrado pela grande quantidade de citações a ele na literatura\hl{, além da pesquisa feita pela \textit{DB-Engines}}\footnote{A pesquisa e a metodologia da pesquisa estão respectivamente disponíveis pelos \textit{links} \url{https://db-engines.com/} e \url{https://db-engines.com/en/ranking_definition}. Acessos ocorridos no dia 16 de dezembro de 2020, às 16:18.} \hl{citá-lo como sendo o banco de dados NoSQL mais popular em novembro de 2020.} Vários são os trabalhos que o comparam a outros bancos de dados relacionais ou não relacionais. Os trabalhos de \citeonline{rautmare2016mysql}; \citeonline{seo2017performance}; \citeonline{grover2016mvm}; \citeonline{fatima2016comparison}; e \citeonline{ansari2018performance} fazem uma comparação de desempenho do \textit{MongoDB} com o banco de dados relacional \textit{MySql}. Neles são gerados gráficos e tabelas que medem o tempo de resposta de cada banco de dados, pela quantidade de dados teste usados em operações de leitura e escrita. \hl{ } Tais resultados demonstram um ganho significativo de performance do \textit{MongoDB} em vários casos de uso. Isto demonstra a relevância da ferramenta, principalmente em aplicações com grande quantidade de dados (\textit{Big Data}). Dentre os trabalhos apresentados, \citeonline{fatima2016comparison} realizam testes com dados reais, visando simular um ambiente IoT, na qual existe a captura de uma grande quantidade de dados de sensores. \hl{Os testes realizados também fazem um comparativo} com o banco de dados \textit{VoltBD}, que utiliza um novo modelo de banco de dados chamado de \textit{NewSQL}.

Além \hl{dos trabalhos apresentados}, também pode ser citado o trabalho \hl{de} \citeonline{pandey2020performance}, que realiza um comparativo mais completo que os apresentados anteriormente, analisando \hl{diversos casos} de teste diferentes. Tais testes também foram realizados sobre o \textit{MongoDB} e o \textit{MySql}, porém medindo o tempo de resposta por quantidade de \textit{threads} em execução, \textit{throughput} (operações/segundo) e latência média, em operações de leitura e escrita. Os resultados também demonstram uma vantagem do \textit{MongoDB} em vários cenários. Dentre os trabalhos apresentados anteriormente, comparativos \hl{que envolvem} cenários \hl{com} várias \textit{threads} ou \hl{a realização de operações simultâneas e distintas} também podem ser encontrados em \citeonline{rautmare2016mysql}; \citeonline{fatima2016comparison}; e \citeonline{ansari2018performance}.

Outros trabalhos que realizam um comparativo com outros bancos de dados podem ser encontrados. \citeonline{moreno2016performance} fazem um comparativo com o banco de dados \textit{Oracle}. Já \citeonline{chopade2017mongodb} realizam um comparativo com outro banco de dados NoSQL chamado de \textit{CouchBase}, utilizando dados de imagens como teste.

As afirmativas feitas na introdução deste trabalho sobre a dificuldade na implementação de operações sobre dados normalizados no \textit{MongoDB} em comparação com o SQL, são confirmadas \hl{pelo trabalho desenvolvido por} \citeonline{celesti2019study}. Neste trabalho, \hl{implementações} manual da junção de documentos, são realizadas usando o \textit{MongoDB}. \hl{Nele, a junção de documentos no \textit{MongoDB}, por meio de uma operação equivalente ao \textit{inner join} do SQL, tornou-se, não só mais complexo de ser desenvolvido, como apresentou performance inferior em comparação com um banco relacional}. Tal resultado demonstra a necessidade e importância de \textit{APIs} e \textit{frameworks} tratarem tal operação, a fim de se obter \hl{ganhos de performance} e facilidade no desenvolvimento. Na seção \ref{section: juncao-documentos}, tal operação de junção será explicitada e alternativas à implementação manual feita por \citeonline{celesti2019study} serão demonstradas, por meio do uso de ferramentas externas ao \textit{MongoDB}.

O \textit{survey} de \citeonline{davoudian2018survey}, assim como os trabalhos apresentados, confirmam as afirmativas deste trabalho da importância do NoSQL na aplicação e desenvolvimento de diversas aplicações, cujo o uso do modelo tradicional Relacional deixa a desejar. Em \citeonline{davoudian2018survey} podem ser encontradas mais informações e detalhes sobre o uso de bancos de \textit{NoSQL}, principalmente no escopo de aplicações distribuídas e escaláveis de forma horizontal.

\hl{O \textit{Mongoose}, sendo uma biblioteca que permite a criação de \textit{schemas} no \textit{MongoDB}, foi escolhido para ser alvo de comparação neste trabalho por causa de sua grande relevância, tando na literatura, quanto na sua popularidade no desenvolvimento de aplicações com \textit{MongoDB} em \textit{Node JS}. O quadro} \ref{quad: livros_mongoose} \hl{referencia vários livros que citam o \textit{Mongoose}, demonstrando sua relevância para o desenvolvimento de aplicações com \textit{MongoDB}.}

% \begin{table}[h]
% \caption{Livros que citam o \textit{Mongoose} \label{livros_mongoose}}
% \begin{tabular}{@{}|l|l|@{}}
% \toprule
% \textbf{Título}                                  & \textbf{Autores}                 \\ \midrule
% \textit{MERN Quick Start Guide}                           & \citeonline{wilson2018mern}      \\ \midrule
% \textit{Web development with MongoDB and NodeJs}          & \citeonline{satheesh2015web}     \\ \midrule
% \textit{Building Node. js REST API with TDD Approach}     & \citeonline{pandian2018building} \\ \midrule
% \textit{Full Stack JavaScript}                            & \citeonline{mardan2018intro}     \\ \midrule
% \textit{Mongoose for Application Development}             & \citeonline{holmes2013mongoose}  \\ \midrule
% \textit{Node. js, MongoDB, and AngularJS web development} & \citeonline{dayley2014node}      \\ \midrule
% \textit{JavaScript Frameworks for Modern Web Development} & \citeonline{bin2019mongoose}     \\ \bottomrule
% \end{tabular}
% \end{table}

\begin{quadro}[h]
\caption{Livros que citam o \textit{Mongoose} \label{quad: livros_mongoose}}
\begin{tabular}{|l|l|}
\hline
\def\arraystretch{1.5}
\begin{tabular}[c]{@{}l@{}}\small{\textbf{Título}}\end{tabular} & \begin{tabular}[c]{@{}l@{}}\small{\textbf{Autores}}\end{tabular} \\ \hline
\def\arraystretch{1.5}
\begin{tabular}[c]{@{}l@{}}\small{MERN Quick Start Guide}\end{tabular} & \small{\citeonline{wilson2018mern}} \\ \hline
\def\arraystretch{1.5}
\begin{tabular}[c]{@{}l@{}}\small{Web development with MongoDB and NodeJs}\end{tabular} & \small{\citeonline{satheesh2015web}} \\ \hline
\def\arraystretch{1.5}
\begin{tabular}[c]{@{}l@{}}\small{Building Node. js REST API with TDD Approach}\end{tabular} & \small{\citeonline{pandian2018building}} \\ \hline
\def\arraystretch{1.5}
\begin{tabular}[c]{@{}l@{}}\small{Full Stack JavaScript}\end{tabular} & \small{\citeonline{mardan2018intro}} \\ \hline
\def\arraystretch{1.5}
\begin{tabular}[c]{@{}l@{}}\small{Mongoose for Application Development}\end{tabular} & \small{\citeonline{holmes2013mongoose}} \\ \hline
\def\arraystretch{1.5}
\begin{tabular}[c]{@{}l@{}}\small{Node. js, MongoDB, and AngularJS web development}\end{tabular} & \small{\citeonline{dayley2014node}} \\ \hline
\def\arraystretch{1.5}
\begin{tabular}[c]{@{}l@{}}\small{JavaScript Frameworks for Modern Web Development}\end{tabular} & \small{\citeonline{bin2019mongoose}} \\ \hline
\end{tabular}
\end{quadro}

% \begin{longtable}[]{|l|l|}
% \caption{Livros que citam o \textit{Mongoose} \label{livros_mongoose}}\tabularnewline
% \toprule
% \textbf{Título} & \textbf{Autores}\tabularnewline
% \midrule
% \endfirsthead
% \toprule
% \textbf{Título} & \textbf{Autores}\tabularnewline
% \midrule
% \endhead
% MERN Quick Start Guide & \citeonline{wilson2018mern}\tabularnewline\hline%\addlinespace
% Web development with MongoDB and NodeJs & \citeonline{satheesh2015web}\tabularnewline\hline%\addlinespace
% Building Node. js REST API with TDD Approach & \citeonline{pandian2018building}\tabularnewline\hline%\addlinespace
% Full Stack JavaScript & \citeonline{mardan2018intro}\tabularnewline\hline%\addlinespace
% Mongoose for Application Development & \citeonline{holmes2013mongoose}\tabularnewline\hline%\addlinespace
% Node. js, MongoDB, and AngularJS web development & \citeonline{dayley2014node}\tabularnewline\hline%\addlinespace
% JavaScript Frameworks for Modern Web Development & \citeonline{bin2019mongoose}\tabularnewline
% \bottomrule
% % \caption*{Fonte: O autor (2020)}
% \end{longtable}
\chapter{\textit{METODOLOGIA}}
\label{Metodologia}

Assim como descrito anteriormente, o foco do \textit{MongoDB} é o armazenamento de dados desnormalizados, disponibilizando diversas funcionalidades para a sua manipulação. Por essa razão, operações de busca e união de documentos normalizados precisam, as vezes, serem implementadas de forma menos intuitiva e mais trabalhosa, em comparação com o SQL. Visando amenizar os problemas apresentados, foi desenvolvido um framework denominado de \textit{Alpha Restful}. Tal ferramenta foi projetada para a linguagem \textit{JavaScript}, usando o ambiente de execução \textit{Node JS}, utilizando internamente o \textit{MongoDB} e o \textit{Mongoose} (uma biblioteca que implementa algumas funcionalidades extras para o \textit{MongoDB}).

O \textit{Alpha Restful} facilita e automatiza diversas funcionalidades para o desenvolvimento de aplicações WEB com \textit{MongoDB}. O presente trabalho tem a proposta de apresentar este framework como uma solução para alguns problemas que envolvem a manipulação de dados normalizados com o \textit{MongoDB}. Por esta razão, dentre todas as funcionalidades disponíves no framework, foram selecionadas 5 delas (Junção de Documentos, Buscas filtradas em Documentos Relacionados, Remoção em Cascata de Documentos Relacionados, Relação de Dependência Entre os Documentos e Identificadores Apontando para Lixo) que melhoram o processo de desenvolvimento de aplicações, dentro do escopo de manipulação de dados normalizados presentes no banco. Outras funcionalidades, mais relacionadas ao desenvolvimento WEB e não tão relacionadas ao \textit{MongoDB} em sí, serão omitidas neste documento, por fugir do escopo principal proposto.

O \textit{Alpha Restful} implementa tais funcionalidades utilizando um conceito apelidado de ``sincronizaçao'', representado pelo objeto de sincronização (\textit{sync}). A sincronização é o estabelecimento de uma conexão lógica entre um atributo de uma entidade com outra entidade. Uma das formas de se estabelecer essa conexão é através do armazenamento de identificadores de outros documentos. Isso seria equivalente ao uso de \textit{foreign key} no modelo relacional. Outras formas de se estabelecer essa conexão não serão explicitadas neste trabalho. Uma sincronização pode ocorrer, tanto na modelagem das entidades, quanto de forma dinâmica, antes ou depois de uma pesquisa.

Uma vez que duas entidades estão sincronizadas, diversas funções podem ser definidas. Cada uma delas acionam eventos que esperam uma das duas entidades serem chamadas por algum método presente no framework. Tanto atributos reais (presentes no banco de dados) quanto atributos dinâmicos (não presentes no banco de dados, pois é criado dinamicamente em memória) podem ser alvo de uma sincronização. Um exemplo de atributo dinâmico será explicitado na seção 4.1.5.1, quando for explicado o funcionamento do relacionamento inverso. Esses atributos dinâmicos são tratados como se eles existissem dentro do banco de dados, sendo possível utilizá-los em pesquisas. Esse tipo de atributo pode ser definido, tanto na modelagem da entidade, quanto dinamicamente na chamada de uma pesquisa ou junção de documentos.

A fim de demonstrar e justificar a importância da ferramenta desenvolvida, bem como explicitar de que forma os problemas apresentados foram solucionados ou amenizados, a seção de ``RESULTADOS'' irá apresentar um comparativo de como cada uma das 5 funcionalidades escolhidas podem ser implementadas com outras ferramentas já disponíveis no mercado, além de demonstrar quais os problemas ou dificuldades são enfrentadas usando tais ferramentas e como a utilização do \textit{Alpha Restful} pode resolver ou amenizar essas problemáticas. A fim de facilitar a apresentação do comparativo, na próxima seção será mostrado um conjunto de documentos exemplo que serão usados como base para explicar como cada funcionalidade pode manipular esses dados.

%diminuir as frases

Para as duas primeiras funcionalidades (Junção de Documentos e Buscas Filtradas em Documentos Relacionados), será explicado, com a demonstração de exemplos práticos do código fonte, como outras ferramentas, presentes no \textit{MongoDB} ou no \textit{Mongoose}, podem ser utilizadas, bem como as vantagens e desvantagens de cada abordagem, demonstrando quais problemas são persistidos em cada uma delas. Após essa demonstração, será apresentado (também com a utilização de exemplos práticos de código fonte) como essa mesma função pode ser feita utilizando o \textit{Alpha Restful}, mostrando como o framework resolve ou ameniza os problemas apresentados anteriormente, bem como quais opções a mais estão disponíveis por meio dessa ferramenta. Para as demais funcionalidades, é apresentado uma breve descrição de como elas podem manualmente ser implementadas usando o \textit{MongoDB} e logo após é apresentado como é possível, através do \textit{Alpha Restful}, resolver tal necessidade de maneira mais automatizada.
\chapter{RESULTADOS}
\label{Resultados}

O \textit{Alpha Restful} facilita e automatiza a implementação de 5 funcionalidades que,
muitas vezes,
%por causa dos motivos descritos\footnote{Emanuel: referenciar a seção} anteriormente,
são complexas de serem desenvolvidas no \textit{MongoDB}, sem o uso de um \textit{framework}. Para cada funcionalidade, será explicado sua implementação sem o uso do \textit{Alpha Restful} e, posteriormente, com o uso deste \textit{framework}. Para o detalhamento destas funcionalidades, será utilizado como base a modelagem de dados exemplificada pelos documentos exemplo \ref{lst: doc-casa-1b}, \ref{lst: doc-casa-1089}, \ref{lst: doc-pessoa-eduardo} e \ref{lst: doc-pessoa-emanuel}.

\begin{lstlisting}[language=json, caption={Documento da Casa de Número 1B\label{lst: doc-casa-1b}}]
{
    "_id": 10,
    "rua": "Rua Castelo branco",
    "numero": "1B"
}
\end{lstlisting}

% \newpage

\begin{lstlisting}[language=json, caption={Documento da Casa de Número 1089\label{lst: doc-casa-1089}}]
{
    "_id": 20,
    "rua": "Rua Pompel",
    "numero": "1089"
}
\end{lstlisting}

% \newpage

\begin{lstlisting}[language=json, caption={Documento da Pessoa \textit{Eduardo}\label{lst: doc-pessoa-eduardo}}]
{
    "_id": 2,
    "nome": "Eduardo",
    "idade": 40,
    "casas": [{
        "_id": ObjectId("5e66cb581766a2056c48145f"),
        "id": 20,
        "valorMensal": 300,
        "quantidadeMeses": 8
    }]
}
\end{lstlisting}

\newpage

\begin{lstlisting}[language=json, caption={Documento da Pessoa \textit{Emanuel}\label{lst: doc-pessoa-emanuel}}]
{
    "_id": 1,
    "nome": "Emanuel",
    "idade": 21,
    "casas": [
        {
            "_id": ObjectId("5e66cb3a7216361ff05b3b8f"),
            "id": 10,
            "valorMensal": 200,
            "quantidadeMeses": 12
        },
        {
            "_id": ObjectId("5e66cb43534f8a1944cdb028"),
            "id ": 20,
            "valorMensal": 400,
            "quantidadeMeses": 24
        }
    ]
}
\end{lstlisting}

\section{Junção de Documentos\label{section: juncao-documentos}}

Assim como foi visto na seção \ref{subsection: juncao-documentos}, as vezes, documentos que foram normalizados precisam ser unidos por meio de seus relacionamentos. A junção de documentos pode ser feita de maneira automática com a utilização do \textit{\$lookup}, \textit{DBRef}, ou do \textit{populate} do \textit{Mongoose}, porém estas abordagens possuem algumas limitações, assim como será demonstrado nas subseções a seguir.

\subsection{\textit{\$lookup}}

O \textit{\$lookup} é uma operação disponibilizada de forma oficial pelo \textit{MongoDB}. Tal operação se responsabiliza por unir dois documentos relacionados. Após tal união, pesquisas podem ser realizadas sobre esses dados e valores podem ser agrupados e ordenados. No exemplo \ref{lst: juncao-lookup-com-omissao} é apresentado uma codificação, utilizando o \textit{\$lookup}, para unir as coleções de pessoas e casas, ignorando os atributos de relacionamento.

\newpage

\begin{lstlisting}[style=ES6, caption={Junção de Documentos com Omissão\label{lst: juncao-lookup-com-omissao}}]
    let UNIAO = await db.collection("pessoas").aggregate([
        { $lookup: {
            from: "casas",
            localField: "casas.id",
            foreignField: "_id",
            as: "casas"
        }}
    ]).toArray()
\end{lstlisting}

A opção ``\textit{from}'' contém o nome da coleção de documentos relacionada com a coleção de pessoas. O ``\textit{localField}'' contém o nome do atributo que possui o identificador da entidade relacionada, contido no documento da coleção de pessoas. O ``\textit{foreignField}'' contém o nome do atributo que possui o identificador da entidade relacionada, contido no documento da coleção de casas. A opção ``\textit{as}'' possui o nome do atributo que conterá todos os atributos da entidade relacionada. Após essa operação, a  variável ``UNIAO'' conterá o resultado apresentado pelo exemplo \ref{lst: resultado-juncao-lookup-com-omissao}.

% \newpage

\begin{lstlisting}[language=json, caption={Resultado da Junção com Omissão\label{lst: resultado-juncao-lookup-com-omissao}}]
[{
    "_id": 1,
    "nome": "Emanuel",
    "idade": 21,
    "casas": [
        {
            "_id": 10,
            "rua": "Rua Castelo Branco",
            "numero": "1B"
        },
        {
            "_id": 20,
            "rua": "Rua Pompel",
            "numero": 1089
        }
    ]
},{
    "_id": 2,
    "nome": "Eduardo",
    "idade": 40,
    "casas": [{
        "_id": 20,
        "rua": "Rua Pompel",
        "numero": "1089"
    }]
}]
\end{lstlisting}

No exemplo \ref{lst: resultado-juncao-lookup-com-omissao}, observa-se que os atributos ``valorMensal'' e ``quantidadeMeses'' não estão contidos no resultado da junção. Isso ocorre porque o \textit{\$lookup} sobrescreve o atributo ``casas'' por todos os atributos presentes no documento de Casa. Pode-se observar também que o identificador (\textit{\_id}) da relação é substituída pelo identificador presente na própria entidade. Essa omissão de atributos pode ser um inconveniente, caso seja necessário obter ou realizar operações nos atributos que estão sendo omitidos. Para a utilização do \textit{\$lookup} sem a omissão de tais valores, uma codificação mais complexa e menos intuitiva tornaria-se necessária. No exemplo \ref{lst: juncao-lookup-sem-omissao} é apresentado uma codificação que inclui os atributos anteriormente omitidos.

\begin{lstlisting}[style=ES6, caption={Junção de Documentos sem Omissão\label{lst: juncao-lookup-sem-omissao}}]
    let UNIAO = await db.collection("pessoas").aggregate([
        { $unwind: "$casas" },
        { $lookup: {
            from: "casas",
            let: { casas: "$casas" },
            pipeline: [
                { $match: { $expr: {
                    $eq: [ "$_id", "$$casas.id" ]
                }}},
                { $addFields: {
                    _id: "$$casas._id",
                    id: "$$casas.id",
                    valorMensal: "$$casas.valorMensal",
                    quantidadeMeses: "$$casas.quantidadeMeses"
                }}
            ],
            as: "casas"
        }},
        { $group: {
            _id: {
                _id: "$_id",
                nome: "$nome",
                idade: "$idade"
            },
            casas: { $push: "$casas" }
        }},
        { $project: {
            _id: "$_id._id",
            nome: "$_id.nome",
            idade: "$_id.idade",
            casas: { $reduce: {
                input: "$casas",
                initialValue: [],
                in: { $concatArrays: [ "$$value", "$$this" ] }
            }}
        }}
    ]).toArray()
\end{lstlisting}

Para adicionar os atributos de relacionamento dentro dos objetos de casa, tornou-se necessário utilizar-se de alguns artifícios do \textit{MongoDB}, manipulando a união em baixo nível. Isso é necessário pois os identificadores das casas estão dentro de uma lista. Se uma pessoa pudesse, no máximo, ter uma única casa, uma codificação mais simples poderia ser realizada. Bastaria para isso usar o exemplo de código \ref{lst: juncao-lookup-com-omissao} e apenas colocar na opção \textit{``as''} do \textit{\$lookup}, um outro caminho que não sobrescreveria nenhum atributo já existente. Após a execução do exemplo de código \ref{lst: juncao-lookup-sem-omissao}, o resultado presente no exemplo \ref{lst: resultado-juncao-lookup-sem-omissao} será obtido.

% \newpage

\begin{lstlisting}[language=json, caption={Resultado da Junção de Documentos sem Omissão\label{lst: resultado-juncao-lookup-sem-omissao}}]
[{
    "_id": 1,
    "nome": "Emanuel",
    "idade": 21,
    "casas": [
        {
            "_id": ObjectId("5e66cb3a7216361ff05b3b8f"),
            "id": 10,
            "rua": "Rua Castelo Branco",
            "numero": "1B",
            "valorMensal": 200,
            "quantidadeMeses": 12
        },
        {
            "_id": ObjectId("5e66cb43534f8a1944cdb028"),
            "id": 20,
            "rua": "Rua Pompel",
            "numero": 1089,
            "valorMensal": 400,
            "quantidadeMeses": 24
        }
    ]
},{
    "_id": 2,
    "nome": "Eduardo",
    "idade": 40,
    "casas": {
        "_id": ObjectId("5e66cb581766a2056c48145f"),
        "id": 20,
        "rua": "Rua Pompel",
        "numero": "1089",
        "valorMensal": 300,
        "quantidadeMeses": 8
    }
}]
\end{lstlisting}

Pode-se observar que a implementação feita para obter uma simples junção de documentos pode ser complexa, sendo que tais operações são mais simples e intuitivas usando o SQL, assim como demonstrado no exemplo \ref{lst: uniao-tabelas-sql}, da seção \ref{subsection: juncao-tabelas}. A complexidade aumenta caso deseje-se fazer uniões em cascata, ou seja, unir documentos, que foram unidos com outros documentos. Quanto maior for o nível de uniões a serem feitas, mais complexo o código fica, podendo facilitar a ocorrência de erros humanos de codificação.

\subsection{\textit{DBRef}}

O \textit{DBRef} é um padrão para referenciar outros documentos de outras coleções. Essa convenção tem a finalidade de armazenar o nome da coleção relacionada (\textit{\$ref}), o identificador do documento (\textit{\$id}) e o nome do banco de dados na qual essa coleção está contida (\textit{\$db}). Se o \textit{\$db} não for informado, assume-se que a coleção está presente no banco de dados do documento que o \textit{DBRef} reside. No exemplo a qual está sendo tratado, o \textit{DBRef} pode ser utilizado para as pessoas registradas no sistema se relacionar com suas casas, assim como demonstrado nos exemplos \ref{lst: doc-dbref-eduardo} e \ref{lst: doc-dbref-emanuel}.

\newpage

\begin{lstlisting}[language=json, caption={Documento da Pessoa \textit{Eduardo} com \textit{DBRef}\label{lst: doc-dbref-eduardo}}]
{
    "_id": 2,
    "nome": "Eduardo",
    "idade": 40,
    "casas": [{
        "_id": ObjectId("5e66cb581766a2056c48145f"),
        "$id": 20,
        "$ref": "casas",
        "valorMensal": 300,
        "quantidadeMeses": 8
    }]
}
\end{lstlisting}

% \newpage

\begin{lstlisting}[language=json, caption={Documento da Pessoa \textit{Emanuel} com \textit{DBRef}\label{lst: doc-dbref-emanuel}}]
{
    "_id": 1,
    "nome": "Emanuel",
    "idade": 21,
    "casas": [
        {
            "_id": ObjectId("5e66cb3a7216361ff05b3b8f"),
            "$id": 10,
            "$ref": "casas",
            "valorMensal": 200,
            "quantidadeMeses": 12
        },
        {
            "_id": ObjectId("5e66cb43534f8a1944cdb028"),
            "$id": 20,
            "$ref": "casas",
            "valorMensal": 400,
            "quantidadeMeses": 24
        }
    ]
}
\end{lstlisting}

Pode-se observar que o \textit{DBRef} é utilizado quando o identificador do documento relacionado é armazenado em \textit{\$id} e quando está presente o atributo \textit{\$ref}, contendo o nome da coleção. Essa padronização é utilizada por algumas bibliotecas e \textit{frameworks} para disponibilizar recursos de junção de documentos automáticas. Nesse caso, uniões de uniões de documentos podem ser feitas automaticamente de forma simples, dependendo da ferramenta que está sendo utilizada para o desenvolvimento. Esses recursos provenientes do \textit{DBRef} não está disponível em todas as linguagens, e cada biblioteca ou \textit{framework} pode tratar isso de forma diferente.

Os atributos de relacionamento (``\textit{\_id}'', ``\textit{valorMensal}'', ``\textit{quantidadeMeses}'') não necessariamente são tratados pela biblioteca ou \textit{framework} utilizado, podendo eles serem ignorados. Nesse caso, os dados precisariam ser remodelados para extrair esses atributos para outro lugar. Uma maneira de fazer isso seria criar uma nova coleção de documentos auxiliares. Tais documentos armazenariam os atributos de relacionamento e os identificadores das entidades relacionadas. Depois seria necessário fazer um \textit{DBRef} com o novo documento criado. Caso esses documentos auxiliares sejam aplicados na modelagem exemplo trabalhada nessa seção, os documentos se pareceriam com os exemplos \ref{lst: doc-dbref-auxiliar-eduardo}, \ref{lst: doc-dbref-auxiliar-emanuel}, \ref{lst: doc-dbref-auxiliar-relacionamento-eduardo}, \ref{lst: doc-dbref-auxiliar-relacionamento1-emanuel}, \ref{lst: doc-dbref-auxiliar-relacionamento2-emanuel}, \ref{lst: doc-dbref-auxiliar-casa-1b} e \ref{lst: doc-dbref-auxiliar-casa-1089}.

\begin{lstlisting}[language=json, caption={Documento de \textit{Eduardo} Usando Documento Auxiliar\label{lst: doc-dbref-auxiliar-eduardo}}]
{
    "_id": 2,
    "nome": "Eduardo",
    "idade": 40,
    "casas": [{
        "$id": ObjectId("5e66cb581766a2056c48145f"),
        "$ref": "relacionamento_pessoas_casas"
    }]
}
\end{lstlisting}

\begin{lstlisting}[language=json, caption={Documento de \textit{Emanuel} Usando Documento Auxiliar\label{lst: doc-dbref-auxiliar-emanuel}}]
{
    "_id": 1,
    "nome": "Emanuel",
    "idade": 21,
    "casas": [
        {
            "$id": ObjectId("5e66cb3a7216361ff05b3b8f"),
            "$ref": "relacionamento_pessoas_casas"
        },
        {
            "$id": ObjectId("5e66cb43534f8a1944cdb028"),
            "$ref": "relacionamento_pessoas_casas"
        }
    ]
}
\end{lstlisting}

% \newpage

\begin{lstlisting}[language=json, caption={Relacionamento de \textit{Eduardo} com sua Casa \label{lst: doc-dbref-auxiliar-relacionamento-eduardo}}]
{
    "_id": ObjectId("5e66cb581766a2056c48145f"),
    "valorMensal": 300,
    "quantidadeMeses": 8,
    "casa": {
        "$id": 20,
        "$ref": "casas"
    }
}
\end{lstlisting}

\begin{lstlisting}[language=json, caption={Relacionamento de \textit{Emanuel} Com Sua Primeira Casa \label{lst: doc-dbref-auxiliar-relacionamento1-emanuel}}]
{
    "_id": ObjectId("5e66cb3a7216361ff05b3b8f"),
    "valorMensal": 200,
    "quantidadeMeses": 12,
    "casa": {
        "$id": 10,
        "$ref": "casas"
    }
}
\end{lstlisting}

\begin{lstlisting}[language=json, caption={Relacionamento de \textit{Emanuel} Com Sua Segunda Casa\label{lst: doc-dbref-auxiliar-relacionamento2-emanuel}}]
{
    "_id": ObjectId("5e66cb43534f8a1944cdb028"),
    "valorMensal": 400,
    "quantidadeMeses": 24,
    "casa": {
        "$id": 20,
        "$ref": "casas"
    }
}
\end{lstlisting}

\newpage

\begin{lstlisting}[language=json, caption={Documento da Casa de Número 1B\label{lst: doc-dbref-auxiliar-casa-1b}}]
{
    "_id": 10,
    "rua": "Rua Castelo branco",
    "numero": "1B"
}
\end{lstlisting}

\begin{lstlisting}[language=json, caption={Documento da Casa de Número 1089\label{lst: doc-dbref-auxiliar-casa-1089}}]
{
    "_id": 20,
    "rua": "Rua Pompel",
    "numero": "1089"
}
\end{lstlisting}

Esses documentos auxiliares que representam o relacionamento de pessoa com casa são necessários, caso o \textit{DBRef} esteja sendo utilizado e as ferramentas de desenvolvimento usadas estejam ignorando os atributos de relacionamento. Dependendo das bibliotecas e \textit{frameworks} utilizados, pode ser que esses documentos auxiliares não sejam necessários, e os dados desses documentos possam ser inseridos no documento principal. A abordagem a ser utilizada dependerá da linguagem e da plataforma.. Se o ambiente de execução utilizado suportar tal estratégia, seria possível relacionar as pessoas com suas casas por meio de uma modelagem parecida com os exemplos \ref{lst: doc-dbref-eduardo-sem-documento-auxiliar}, \ref{lst: doc-dbref-emanuel-sem-documento-auxiliar}, \ref{lst: doc-dbref-casa-1b-sem-documento-auxiliar} e \ref{lst: doc-dbref-casa-1089-sem-documento-auxiliar}.

\begin{lstlisting}[language=json, caption={Documento da Pessoa \textit{Eduardo} Sem Documento Auxiliar\label{lst: doc-dbref-eduardo-sem-documento-auxiliar}}]
{
    "_id": 2,
    "nome": "Eduardo",
    "idade": 40,
    "casas": [{
        "_id": ObjectId("5e66cb581766a2056c48145f"),
        "valorMensal": 300,
        "quantidadeMeses": 8,
        "casa": {
            "$id": 20,
            "$ref": "casas"
        }
    }]
}
\end{lstlisting}

\begin{lstlisting}[language=json, caption={Documento da Pessoa \textit{Emanuel} Sem Documento Auxiliar\label{lst: doc-dbref-emanuel-sem-documento-auxiliar}}]
{
    "_id": 1,
    "nome": "Emanuel",
    "idade": 21,
    "casas": [
        {
            "_id": ObjectId("5e66cb3a7216361ff05b3b8f"),
            "valorMensal": 200,
            "quantidadeMeses": 12,
            "casa": {
                "$id": 10,
                "$ref": "casas"
            }
        },
        {
            "_id": ObjectId("5e66cb43534f8a1944cdb028"),
            "valorMensal": 400,
            "quantidadeMeses": 24,
            "casa": {
                "$id": 20,
                "$ref": "casas"
            }
        }
    ]
}
\end{lstlisting}

\begin{lstlisting}[language=json, caption={Documento da Casa de Número 1B\label{lst: doc-dbref-casa-1b-sem-documento-auxiliar}}]
{
    "_id": 10,
    "rua": "Rua Castelo branco",
    "numero": "1B"
}
\end{lstlisting}

% \newpage

\begin{lstlisting}[language=json, caption={Documento da Casa de Número 1089\label{lst: doc-dbref-casa-1089-sem-documento-auxiliar}}]
{
    "_id": 20,
    "rua": "Rua Pompel",
    "numero": "1089"
}
\end{lstlisting}

Essa última abordagem apresentada possui a vantagem de não existir um documento auxiliar intermediando o relacionamento. Isso é útil por diminuir o armazenamento e por permitir que consultas mais complexas sejam mais simples de se implementar.

Assim como explicado anteriormente, o \textit{DBRef} pode permitir que operações de união de documentos ocorram de forma mais simples e automática, caso o ambiente de execução utilizado disponibilize tais opções para os BSONs que seguem seu padrão. Apesar dos benefícios, operações mais complexas como, por exemplo, buscas sobre valores presentes em vários documentos, podem não estar disponíveis via \textit{DBRef}. Além disso, operações como o \textit{\$lookup} podem exigir que o \textit{DBRef} seja convertido para um outro formato antes de tais operações serem realizadas. 

Por causa disso, o uso do \textit{DBRef} pode exigir codificações mais complexas em determinadas situações onde a biblioteca ou \textit{framework} não daria suporte. Por essas razões que, dependendo das ferramentas disponibilizadas pelo ambiente de execução, bem como das necessidades do projeto, pode ser que o uso de uma referencia manual de outros documentos seja uma melhor escolha do que o \textit{DBRef}.

\subsection{\textit{Populate} do \textit{Mongoose}}

O \textit{Mongoose} é uma biblioteca feita para \textit{Node JS}, que disponibiliza algumas funcionalidades a mais para o \textit{MongoDB}, principalmente relacionadas à modelagem dos dados. Através dessa ferramenta, torna-se possível criar \textit{schemas} para os dados a serem armazenados. Esses \textit{schemas} permitem que a estrutura do BSON de cada documento seja padronizada e obedeça as regras de estrutura definidos pelo programador. Para o exemplo que está sendo apresentado, os \textit{schemas} para as coleções de pessoas e casas podem ser definidos, assim como demonstrado no exemplo \ref{lst: schema-mongoose}.

% \newpage

\begin{lstlisting}[style=ES6, caption={Definição de \textit{Schemas} no \textit{Mongoose}\label{lst: schema-mongoose}}]
    const PessoaSchema = new mongoose.Schema({
        _id: Number,
        nome: String,
        idade: Number,
        casas: [{
            _id: mongoose.Schema.Types.ObjectId,
            id: {
                type: Number,
                ref: "casas"
            },
            valorMensal: Number,
            quantidadeMeses: Number
        }]
    })
    const Pessoa = db.model("pessoas", PessoaSchema)

    const CasaSchema = new mongoose.Schema({
        _id: Number,
        rua: String,
        numero: String
    })
    const Casa = db.model("casas", CasaSchema)
\end{lstlisting}

Um dos recursos disponibilizados pelo \textit{Mongoose} é o \textit{populate}. Este recurso substitui o \textit{DBRef} e o \textit{\$lookup} para operações de busca, seguido da junção de documentos. Uma das restrições do \textit{populate}, em comparação com o \textit{\$lookup}, é que no \textit{\$lookup} é possível realizar buscas sobre os dados dos documentos unidos, enquanto que no \textit{populate} a busca deve ocorrer apenas sobre os dados originais de cada documento. Se desconsiderarmos essa limitação, o \textit{populate} é mais simples de se usar do que o \textit{\$lookup}, além de disponibilizar novas opções e recursos. Atualmente, o \textit{Mongoose} está disponível apenas para o \textit{Node JS}, que é exatamente o escopo a qual esse trabalho se propõe em atual. O exemplo \ref{lst: juncao-populate} apresenta a junção das coleções de pessoas e casas usando o \textit{populate}. Com apenas uma única linha, os dois documentos são unidos, mantendo os atributos de relacionamento. Após a execução do código exemplo \ref{lst: juncao-populate}, a variável ``UNIAO'' terá o resultado apresentado pelo exemplo \ref{lst: resultado-juncao-populate}.

\begin{lstlisting}[style=ES6, caption={Junção de Documentos Com o \textit{Populate}\label{lst: juncao-populate}}]
  let UNIAO = await Pessoa.find({}).populate("casas.id").exec()
\end{lstlisting}

\begin{lstlisting}[language=json, caption={Resultado da Junção de Documentos Com o \textit{Populate}\label{lst: resultado-juncao-populate}}]
[{
    "_id": 1,
    "nome": "Emanuel",
    "idade": 21,
    "casas": [
        {
            "_id": ObjectId("5e66cb3a7216361ff05b3b8f"),
            "id": {
                "_id": 10,
                "rua": "Rua Castelo Branco",
                "numero": "1B"
            },
            "valorMensal": 200,
            "quantidadeMeses": 12
        },
        {
            "_id": ObjectId("5e66cb43534f8a1944cdb028"),
            "id": {
                "_id": 10,
                "rua": "Rua Pompel",
                "numero": 1089
            },
            "valorMensal": 400,
            "quantidadeMeses": 24
        }
    ]
},
{
    "_id": 2,
    "nome": "Eduardo",
    "idade": 40,
    "casas": {
        "_id": ObjectId("5e66cb581766a2056c48145f"),
        "id": {
            "_id": 10,
            "rua": "Rua Pompel",
            "numero": 1089
        },
        "valorMensal": 300,
        "quantidadeMeses": 8
    }
}]
\end{lstlisting}

Pode-se observar que os valores de casa foram preenchidos no atributo onde encontra-se o identificador do documento. Por causa dessa característica do \textit{populate}, pode ser mais intuitivo, para este caso, remodelar a entidade de pessoa, para que o atributo ``\textit{id}'' seja renomeado para, por exemplo, ``casa''. Nesse caso, os resultados sairiam com o formato apresentado pelo exemplo \ref{lst: resultado-juncao-populate-id-renomeado}.

\begin{lstlisting}[language=json, caption={Resultado do \textit{Populate} com ``id'' Renomeado para ``casa''\label{lst: resultado-juncao-populate-id-renomeado}}]
[{
    "_id": 1,
    "nome": "Emanuel",
    "idade": 21,
    "casas": [
        {
            "_id": ObjectId("5e66cb3a7216361ff05b3b8f"),
            "casa": {
                "_id": 10,
                "rua": "Rua Castelo Branco",
                "numero": "1B"
            },
            "valorMensal": 200,
            "quantidadeMeses": 12
        },
        {
            "_id": ObjectId("5e66cb43534f8a1944cdb028"),
            "casa": {
                "_id": 10,
                "rua": "Rua Pompel",
                "numero": 1089
            },
            "valorMensal": 400,
            "quantidadeMeses": 24
        }
    ]
},{
    "_id": 2,
    "nome": "Eduardo",
    "idade": 40,
    "casas": {
        "_id": ObjectId("5e66cb581766a2056c48145f"),
        "casa": {
            "_id": 10,
            "rua": "Rua Pompel",
            "numero": 1089
        },
        "valorMensal": 300,
        "quantidadeMeses": 8
    }
}]
\end{lstlisting}

\subsection{Junção de Documentos de Forma Manual}

Caso haja a necessidade de obter um maior controle na junção dos documentos, ou se os recursos de junção automático não forem satisfatório o suficiente para as necessidades da aplicação, é possível unir documentos de forma manual. Uma codificação manual está apresentado no exemplo \ref{lst: juncao-manual}. Ao final do código deste exemplo, a variável ``pessoas'' armazenará o resultado apresentado pelo exemplo \ref{lst: resultado-juncao-manual}.

% \newpage

\begin{lstlisting}[style=ES6, caption={Junção Manual dos Documentos de Pessoa com Casa\label{lst: juncao-manual}}]
    let pessoas = await db.collection("pessoas")
        .find({}).toArray();
    
    for (let p of pessoas) {
        for (let i = 0; i < p.casas.length; i++) {
            let c = p.casas[i];
    
            let casa = (await db.collection("casas").find({
                "_id": c.id
            }).toArray())[0];
    
            p.casas[i] = {
                ...casa,
                ...c
            };
        }
    }
\end{lstlisting}

% \newpage

\begin{lstlisting}[language=json, caption={Junção de Documentos de Forma Manual\label{lst: resultado-juncao-manual}}]
[{
    "_id": 1,
    "nome": "Emanuel",
    "idade": 21,
    "casas": [
        {
            "_id": ObjectId("5e66cb3a7216361ff05b3b8f"),
            "id": 10,
            "rua": "Rua Castelo Branco",
            "numero": "1B",
            "valorMensal": 200,
            "quantidadeMeses": 12
        },
        {
            "_id": ObjectId("5e66cb43534f8a1944cdb028"),
            "id": 20,
            "rua": "Rua Pompel",
            "numero": 1089,
            "valorMensal": 400,
            "quantidadeMeses": 24
        }
    ]
},
{
    "_id": 2,
    "nome": "Eduardo",
    "idade": 40,
    "casas": {
        "_id": ObjectId("5e66cb581766a2056c48145f"),
        "id": 20,
        "rua": "Rua Pompel",
        "numero": "1089",
        "valorMensal": 300,
        "quantidadeMeses": 8
    }
}]
\end{lstlisting}

%Observa-se que em um ambiente real, vários documentos %precisariam ser unidos, deixando o código cada vez mais complexo.

\subsection{Usando o \textit{Alpha Restful} Para Unir Documentos}

O \textit{Alpha Restful} possui sua própria implementação para unir os documentos. Internamente, os documentos sempre são unidos de forma manual. A vantagem de se utilizar o \textit{Alpha Restful} é que ele disponibiliza duas novas funcionalidades que não estão diretamente disponíveis pelos métodos descritos anteriormente:

\begin{itemize}
	\item Relacionamento inverso
	\item Relacionamento inverso de relacionamento inverso
\end{itemize}

Para que as funcionalidades do \textit{framework} sejam disponibilizadas, o \textit{Alpha Restful} utiliza os \textit{schemas} do \textit{Mongoose}, em conjunto com especificações de sincronização (\textit{sync}) entre as entidades. Essas especificações de sincronização permitem que entidades sejam relacionadas entres si baseado em identificadores e em outros tipos de relacionamentos (que não serão explicados por fugir do escopo desse trabalho). O exemplo \ref{lst: schemas-sync-alpha-restful} apresenta uma codificação para criar os \textit{schemas} e especificações de sincronização das entidades Pessoa e Casa, usando o \textit{Alpha Restful}.

% \newpage

\begin{lstlisting}[style=ES6, caption={Definição de \textit{Schemas} no \textit{Alpha Restful}\label{lst: schemas-sync-alpha-restful}}]
    const restful = new Restful(
        "<nome-da-aplicacao>",
        { locale: "pt" }
    )

    const Pessoa = new Entity({
        name: "Pessoa",
        resource: "pessoas",
        descriptor: {
            nome: String,
            idade: Number,
            casas: [{
                id: Number,
                valorMensal: Number,
                quantidadeMeses: Number
            }]
        },
        sync: {
            casas: {
                name: "Casa",
                fill: true
            }
        }
    })
    
    const Casa = new Entity({
        name: "Casa",
        resource: "casas",
        descriptor: {
            rua: String,
            numero: String
        }
    })
    
    restful.add(Pessoa)
    restful.add(Casa)
\end{lstlisting}

Para a junção de documentos, o \textit{Alpha Restful} disponibiliza uma opção denominada de \textit{fill}. Tal opção é poderosa, pois ela pode ser utilizada sobre qualquer objeto já pesquisado ou montado, sendo possível definir, na chamada da função, relacionamentos temporários com outras entidades. O exemplo \ref{lst: juncao-alpha-restful} realiza uma junção de documentos sobre as pessoas e casas, usando a modelagem definida em \ref{lst: schemas-sync-alpha-restful}.

\begin{lstlisting}[style=ES6, caption={Junção de Documentos Com o \textit{Alpha Restful}\label{lst: juncao-alpha-restful}}]
	let pessoas = await Pessoa.model.find({}).exec()
	pessoas = await Pessoa.fill(pessoas, restful)
\end{lstlisting}

A operação de junção de documentos ocorre após os objetos terem sido obtidos, por exemplo, por meio de uma busca. Nesse caso buscou-se por todas as pessoas. Após essa busca, basta chamar o método \textit{Pessoa.fill} para que os documentos sejam unidos. Como na modelagem já tinha sido definido que o atributo ``casas'' de pessoa fazia um relacionamento com a entidade ``Casa'' (através do objeto \textit{sync}), e que por padrão os documentos devem ser unidos (através da opção \textit{fill} igual a \textit{true} em \textit{sync}), apenas a chamada do método é o suficiente para realizar a junção. Na versão atual do \textit{Alpha Restful} (0.7.38), o identificador da entidade relacionada precisa ser definido no atributo \textit{id}. Depois de executar o código \ref{lst: juncao-alpha-restful}, a variável ``pessoas'' obterá o resultado apresentado pelo exemplo \ref{lst: resultado-juncao-alpha-restful}.

% \newpage

\begin{lstlisting}[language=json, caption={Resultado da Junção de Documentos Com o \textit{Alpha Restful}\label{lst: resultado-juncao-alpha-restful}}]
[{
    "_id": 1,
    "nome": "Emanuel",
    "idade": 21,
    "casas": [
        {
            "_id": ObjectId("5e66cb3a7216361ff05b3b8f"),
            "id": 10,
            "rua": "Rua Castelo Branco",
            "numero": "1B",
            "valorMensal": 200,
            "quantidadeMeses": 12
        },
        {
            "_id": ObjectId("5e66cb43534f8a1944cdb028"),
            "id": 20,
            "rua": "Rua Pompel",
            "numero": 1089,
            "valorMensal": 400,
            "quantidadeMeses": 24
        }
    ]
},
{
    "_id": 2,
    "nome": "Eduardo",
    "idade": 40,
    "casas": {
        "_id": ObjectId("5e66cb581766a2056c48145f"),
        "id": 20,
        "rua": "Rua Pompel",
        "numero": "1089",
        "valorMensal": 300,
        "quantidadeMeses": 8
    }
}]
\end{lstlisting}

\subsubsection{Relacionamento Inverso\label{subsubsection: relacionamento-inverso}}

Uma das funcionalidades disponibilizadas pelo \textit{Alpha Restful} que, atualmente, não estão diretamente disponíveis nos outros métodos de junção de documentos mostrados (com exceção do método manual), é o relacionamento inverso. Para ilustrar tal funcionalidade, pode-se analisar a situação onde seria necessário unir as duas coleções, mas juntando nos documentos de casa. Essa operação pode ser feita diretamente no método de \textit{fill}, sem a necessidade de alterar a modelagem da entidade. O exemplo \ref{lst: juncao-alpha-restful-relacionamento-inverso} apresenta uma codificação que realiza tal junção, definindo o relacionamento na entidade Casa, diretamente pelo método \textit{Casa.fill}, sem alterar a modelagem apresentada no exemplo \ref{lst: schemas-sync-alpha-restful}.

\newpage

\begin{lstlisting}[style=ES6, caption={Junção de Documentos em Relacionamento Inverso\label{lst: juncao-alpha-restful-relacionamento-inverso}}]
    let casas = await Casa.model.find({}).exec()
    casas = await Casa.fill(casas, restful, { sync: {
        pessoas: {
            name: "Pessoa",
            syncronized: ["casas"],
            fill: true,
            jsonIgnoreProperties: "casas"
        }
    }})
\end{lstlisting}

A opção ``\textit{jsonIgnoreProperties}'', nesse caso, é responsável por ignorar o atributo ``casas'' de Pessoa. Isso é necessário para que não ocorra uma recursão infinita. Sem essa opção, as pessoas seriam preenchidas no atributo ``pessoas'', as casas seriam preenchidas no atributo ``casas'', as pessoas seriam novamente preenchidas no atributo ``pessoas'' e assim por diante. Com a opção ``\textit{jsonIgnoreProperties}'', as casas não serão incluídas nas pessoas. Após a execução do código exemplo \ref{lst: juncao-alpha-restful-relacionamento-inverso}, as coleções são unidas, mas tendo como base a entidade Casa. Os documentos unidos estarão presentes na variável ``casas'' e terá a estrutura definida pelo exemplo \ref{lst: resultado-juncao-alpha-restful-relacionamento-inverso}.

% \newpage

\begin{lstlisting}[language=json, caption={Resultado da Junção de Documentos em ``Casa''\label{lst: resultado-juncao-alpha-restful-relacionamento-inverso}}]
[
    {
        "_id": 10,
        "rua": "Rua Castelo Branco",
        "numero": "1B",
        "pessoas": [
            {
                "id": 1,
                "nome": "Emanuel",
                "idade": 21
            }
        ]
    },
    {
        "_id": 20,
        "rua": "Rua Pompel",
        "numero": "1089",
        "pessoas": [
            {
                "id": 1,
                "nome": "Emanuel",
                "idade": 21
            },
            {
                "id": 2,
                "nome": "Eduardo",
                "idade": 40
            }
        ]
    }
]
\end{lstlisting}

Como na modelagem da entidade Casa o relacionamento com pessoa não foi definido, é possível fazer essa definição na hora de realizar a junção. Pode-se observar que não existe a necessidade de armazenar os identificadores das pessoas nas suas casas, o \textit{framework} consegue automaticamente identificálos, bastando apenas informar na opção ``\textit{syncronized}'' o caminho para se obter as casas por meio de uma pessoa. Caso fosse desejado obter esse comportamento por padrão, assim como ocorre em ``Pessoa'', bastaria atualizar o objeto ``\textit{sync}'' da entidade ``Casa''. Se isso for feito, a junção poderá ocorrer sem a definição do relacionamento na hora de chamar o método de junção. Neste caso, a modelagem de Casa seria definido, assim como apresentado no exemplo \ref{lst: schemas-sync-alpha-restful-relacionamento-inverso}. Se a modelagem for definida desta maneira, a junção poderá ser executada, assim como demonstrado no exemplo \ref{lst: resultado-juncao-schemas-sync-alpha-restful-relacionamento-inverso}.

\begin{lstlisting}[style=ES6, caption={Relacionamento Inverso em \textit{Schema} de Casa\label{lst: schemas-sync-alpha-restful-relacionamento-inverso}}]
    const Casa = new Entity({
        name: "Casa",
        resource: "casas",
        descriptor: {
            rua: String,
            numero: String
        },
        sync: {
            pessoas: {
                name: "Pessoa",
                syncronized: ["casas"],
                fill: true,
                jsonIgnoreProperties: "casas"
            }
        }
    })
\end{lstlisting}

\begin{lstlisting}[style=ES6, caption={Junção Com o \textit{Alpha Restful} em Relacionamento Inverso\label{lst: resultado-juncao-schemas-sync-alpha-restful-relacionamento-inverso}}]
	let casas = await Casa.model.find({}).exec()
	casas = await Casa.fill(casas, restful)
\end{lstlisting}

\subsubsection{Relacionamento Transitivo}

Outra funcionalidade disponibilizada pelo \textit{Alpha Restful} que, atualmente, não está disponível nos outros métodos de junção de documentos mostrados (com exceção do método manual), é o relacionamento transitivo. Uma forma de ilustrar esta função é tentar obter, no documento das pessoas, a lista de moradores de uma ou mais casas que a própria pessoa também mora. Para isto, bastaria fazer um relacionamento do atributo ``pessoas'' em casas, assim como definido no exemplo \ref{lst: schemas-sync-alpha-restful-relacionamento-inverso-de-relacionamento-inverso}.

\begin{lstlisting}[style=ES6, caption={Relacionamento Inverso de Relacionamento Inverso \label{lst: schemas-sync-alpha-restful-relacionamento-inverso-de-relacionamento-inverso}}]
    const restful = new Restful("<nome-da-aplicacao>", {
        locale: "pt"
    })

    const Pessoa = new Entity({
        name: "Pessoa",
        resource: "pessoas",
        descriptor: {
            nome: String,
            idade: Number,
            casas: [{
                id: Number,
                valorMensal: Number,
                quantidadeMeses: Number
            }]
        },
        sync: {
            casas: {
                name: "Casa",
                fill: true,
                jsonIgnoreProperties: "pessoas"
            },
            residentes: {
                name: "Pessoa",
                syncronized: ["casas.pessoas"],
                fill: true,
                jsonIgnoreProperties: ["residentes", "casas"]
            }
        }
    })
    
    const Casa = new Entity({
        name: "Casa",
        resource: "casas",
        descriptor: {
            rua: String,
            numero: String
        },
        sync: {
            pessoas: {
                name: "Pessoa",
                syncronized: ["casas"],
                fill: true,
                jsonIgnoreProperties: ["casas", "residentes"]
            }
        }
    })
    
    restful.add(Pessoa)
    restful.add(Casa)
\end{lstlisting}

Os \textit{schemas} definidos no exemplo \ref{lst: schemas-sync-alpha-restful-relacionamento-inverso-de-relacionamento-inverso} criam, nos documentos de pessoas, um novo atributo (``residentes'') que contém todas as pessoas que então contidas no atributo definido em \textit{``syncronized''} que, neste caso, é o atributo ``casas.pessoas''. Pode-se observar a presença da opção ``\textit{jsonIgnoreProperties}'' no ``\textit{sync}'' das entidades. Tal opção armazena o nome do atributo (poderia ser uma lista de nomes de atributos) que será ignorado nos documentos que serão unidos. Isso é necessário para impedir uma união recursiva infinita de atributos. Da forma como a modelagem está definida, ao unir os documentos de pessoas e casas, haverá um atributo nos documentos de pessoas chamado de ``residentes''. Este, conterá todas as pessoas que moram em uma ou mais casas na qual a própria pessoa também mora. Essa união de documentos pode ser realizado por meio do código exemplo \ref{lst: juncao-alpha-restful-relacionamento-inverso-do-relacionamento-inverso}.

\begin{lstlisting}[style=ES6, caption={Junção de Documentos Com o \textit{Alpha Restful}\label{lst: juncao-alpha-restful-relacionamento-inverso-do-relacionamento-inverso}}]
	let pessoas = await Pessoa.model.find({}).exec()
	pessoas = await Pessoa.fill(pessoas, restful)
\end{lstlisting}

Por causa do relacionamento inverso, que também pode se relacionar com um outro relacionamento inverso, a união de documentos por meio do \textit{Alpha Restful} é mais poderosa que as outras opções descritas anteriormente. Ao final da execução do código exemplo \ref{lst: juncao-alpha-restful-relacionamento-inverso-do-relacionamento-inverso}, a variável ``pessoas'' terá o resultado do exemplo \ref{lst: resultado-juncao-alpha-restful-relacionamento-inverso-do-relacionamento-inverso}.

\begin{lstlisting}[language=json, caption={Resultado da Junção de Documentos Com Residentes\label{lst: resultado-juncao-alpha-restful-relacionamento-inverso-do-relacionamento-inverso}}]
[{
    "_id": 1,
    "nome": "Emanuel",
    "idade": 21,
    "casas": [
        {
            "_id": ObjectId("5e66cb3a7216361ff05b3b8f"),
            "id": 10,
            "rua": "Rua Castelo Branco",
            "numero": "1B",
            "valorMensal": 200,
            "quantidadeMeses": 12
        },
        {
            "_id": ObjectId("5e66cb43534f8a1944cdb028"),
            "id": 20,
            "rua": "Rua Pompel",
            "numero": 1089,
            "valorMensal": 400,
            "quantidadeMeses": 24
        }
    ],
    "residentes": [
        {
            "id": 1,
            "nome": "Emanuel",
            "idade": 21,
        },
        {
            "id": 2,
            "nome": "Eduardo",
            "idade": 40,
        }
    ]
},
{
    "_id": 2,
    "nome": "Eduardo",
    "idade": 40,
    "casas": [{
        "_id": ObjectId("5e66cb581766a2056c48145f"),
        "id": 20,
        "rua": "Rua Pompel",
        "numero": "1089",
        "valorMensal": 300,
        "quantidadeMeses": 8
    }],
    "residentes": [
        {
            "id": 1,
            "nome": "Emanuel",
            "idade": 21,
        },
        {
            "id": 2,
            "nome": "Eduardo",
            "idade": 40,
        }
    ]
}]
\end{lstlisting}

\subsection{Junção de Documentos equivalente ao \textit{inner join} do SQL\label{subsection: inner-join-mongodb}}

Os exemplos de código para junção de documentos apresentados anteriormente são equivalentes ao \textit{left join} do SQL. Isto significa que a entidade sobre a qual a busca é feita, é obtida, mesmo que não haja nenhum documento a ser unido com ela. Caso seja necessário haver uma junção na qual deve ser obtido apenas as entidades que possuem documentos a serem unidos, ou seja, uma junção equivalente ao \textit{inner join} do SQL, pode-se filtrar apenas as entidades que possuem um identificador de outro documento. No exemplo \ref{lst: juncao-alpha-restful}, fazer uma junção equivalente ao \textit{inner join} obteria apenas as pessoas que possuem casas. Isto pode ser implementado, assim como demonstrado no exemplo \ref{lst: inner-join-alpha-restful}, usando o \textit{Alpha Restful}.

\begin{lstlisting}[style=ES6, caption={\textit{inner join} de Documentos Com o \textit{Alpha Restful\label{lst: inner-join-alpha-restful}}}]
	let pessoas = await Pessoa.model.find({
	    "casas.id": { $ne: null }
	}).exec()
	pessoas = await Pessoa.fill(pessoas, restful)
\end{lstlisting}

\section{Buscas Filtradas em Documentos Relacionados\label{section: buscas-filtradas-documentos-relacionados}}

O \textit{MongoDB} é nativamente capaz de realizar buscas complexas e simples dentro de um mesmo documento. Porém, a partir do momento que as buscas precisam ser realizadas em documentos normalizados, as pesquisas passam a ficar mais difíceis de serem realizadas. No exemplo \ref{lst: busca-simples-mongodb} é apresentado um código que realiza uma busca pelas pessoas cuja a idade seja igual a 40.

% \newpage

\begin{lstlisting}[style=ES6, caption={Busca de Pessoas com Idade Igual a 40\label{lst: busca-simples-mongodb}}]
    let pessoas = await db.collection("pessoas").find({
        "idade": 40
    }).toArray()
\end{lstlisting}

Ao final desse código, a variável ``pessoas'' obterá todas as pessoas na qual a idade é igual a 40. O código é simples, pois a busca ocorre dentro do próprio documento. Mas e se for desejado obter todas as casas, na qual existe pelo menos uma pessoa, que essa pessoa possui pelo menos uma casa, que nessa casa possui pelo menos uma pessoa que possui a idade igual a 40 anos?

\subsection{Pesquisa Usando o \textit{\$lookup} e \textit{\$match}\label{subsection: pesquisa-lookup-match}}

Para realizar a pesquisa proposta, uma abordagem possível é unir os documentos utilizando o \textit{\$lookup}, até o nível que todos os dados estejam no mesmo documento. Após essa união, torna-se possível fazer a pesquisa, utilizando o parâmetro ``\textit{\$match}''. No exemplo \ref{lst: busca-lookup-match} é apresentado uma implementação da busca proposta, utilizando \textit{\$lookup} e \textit{\$match}.

\begin{lstlisting}[style=ES6, caption={Busca em Dados Normalizados Com o \textit{\$lookup}\label{lst: busca-lookup-match}}]
    let RESULTADO = await db.collection("casas").aggregate([
        { $lookup: {
            from: "pessoas",
            localField: "_id",
            foreignField: "casas.id",
            as: "pessoas"
        }},
        { $unwind: "$pessoas" },
        { $lookup: {
            from: "casas",
            localField: "pessoas.casas.id",
            foreignField: "_id",
            as: "pessoas.casas"
        }},
        { $unwind: "$pessoas.casas" },
        { $lookup: {
            from: "pessoas",
            localField: "pessoas.casas._id",
            foreignField: "casas.id",
            as: "pessoas.casas.pessoas"
        }},
        { $group: {
            _id: {
                _id: "$_id",
                rua: "$rua",
                numero: "$numero",
                pessoas: {
                    _id: "$pessoas._id",
                    nome: "$pessoas.nome",
                    idade: "$pessoas.idade",
                }
            },
            casas: {
                $push: "$pessoas.casas"
            }
        }},
        { $project: {
            _id: "$_id._id",
            rua: "$_id.rua",
            numero: "$_id.numero",
            pessoas: {
                _id: "$_id.pessoas._id",
                nome: "$_id.pessoas.nome",
                idade: "$_id.pessoas.idade",
                casas: "$casas"
            }
        }},
        { $group: {
            _id: {
                _id: "$_id",
                rua: "$rua",
                numero: "$numero"
            },
            pessoas: {
                $push: "$pessoas"
            }
        }},
        { $project: {
            _id: "$_id._id",
            rua: "$_id.rua",
            numero: "$_id.numero",
            pessoas: "$pessoas"
        }},
        { $match: {
            "pessoas.casas.pessoas.idade": 40
        }}
    ]).toArray()
\end{lstlisting}

Para a realização de uma pesquisa dessa complexidade, é necessário unir os documentos várias vezes, pois é preciso acessar os dados que estão dentro de uma lista (pessoas), que estão dentro de uma lista (pessoas.casas), que estão dentro de uma lista (pessoas.casas.pessoas). Por essa razão, utilizar o \textit{\$lookup} para consultas pode ser complexo.

\subsection{Pesquisa Manual\label{subsection: pesquisa-manual}}
    
Também é possível realizar a pesquisa proposta de forma manual. Para isso, pode-se subdividir a pesquisa em pesquisas menores. Logo após, basta uni-las em uma pesquisa que irá obter o resultado esperado. O exemplo \ref{lst: busca-manual} apresenta uma codificação utilizando esta abordagem, sendo necessário 30 linhas de código.

%\newpage

%Através de uma pesquisa pela internet, é comumente encontrado, na maioria dos fóruns pesquisados, que a pesquisa em vários documentos relacionados deve ser quebrado em várias pesquisas em cada documento. 

\begin{lstlisting}[style=ES6, caption={Busca em Dados Normalizados de Forma Manual\label{lst: busca-manual}}]
    let pessoasIdade40 = await db.collection("pessoas").find({
        "idade": 40
    }).toArray();
    
    let idsCasasPessoasIdade40 = 
    pessoasIdade40.map(p => 
        p.casas.reduce((a,c) => [...a,c.id], [])
    ).reduce((a,lid) => [...a, ...lid], []);
    
    let casasPessoasIdade40 =
    await db.collection("casas").find({
        "_id": { $in: idsCasasPessoasIdade40 }
    }).toArray();
    
    let idsCasasPessoasIdade40 = 
    casasPessoasIdade40.map(c => c._id);
    
    let pessoasCasasPessoasIdade40 = await
    db.collection("pessoas").find({
        "casas.id": { $in: idsCasasPessoasIdade40 }
    }).toArray();
    
    let idsCasasPessoasCasasPessoasIdade40 = 
    idsCasasPessoasCasasPessoasIdade40.map(p => 
        p.casas.reduce((a,c) => [...a,c.id], [])
    ).reduce((a,lid) => [...a, ...lid], []);
    
    let RESULTADO_DA_PESQUISA = await
    db.collection("casas").find({
        "_id": { $in: idsCasasPessoasCasasPessoasIdade40 }
    }).toArray();
\end{lstlisting}
    
O primeiro passo para realizar esta busca é de obter todas as pessoas que possui idade igual a 40 anos (linha 1 a 3). Depois, os identificadores das casas pertencentes a estas pessoas são extraídos (linhas 5 a 8). Após a extração destes identificadores, são buscados todas as casas que possui um identificador dentre esses (linhas 10 a 12). Após a busca de todas essas casas, são extraídos todos os identificares (linhas 14 a 15). Após a extração desses identificadores, são buscadas todas as pessoas que possuem alguma destas casas (linhas 17 a 20). Após a busca destas pessoas, são extraídos todos os identificadores das casas destas pessoas (linhas 22 a 25). Finalmente, as casas que possui seu identificador dentre os identificadores são buscadas, obtendo o resultado esperado pela consulta. Observa-se que, tanto esta consulta, quando a consulta usando o \textit{\$lookup} e \textit{\$match}, ficariam mais complexas com a adição de outros filtros, utilizando-se de outras relações com outros documentos relacionados.

\subsection{Pesquisa Usando o \textit{Alpha Restful}\label{subsection: pesquisa-usando-alpha-restful}}

Como observado nas seções \ref{subsection: pesquisa-lookup-match} e \ref{subsection: pesquisa-manual}, realizar buscas que englobam vários documentos pode ser algo complexo. Para contornar este problema, o \textit{Alpha Restful} mapeia todos os relacionamentos normalizados dentre todos os documentos do sistema para fornecer uma sintaxe simples para se realizar pesquisas. Para ilustrar isto, é demonstrado no exemplo \ref{lst: busca-alpha-restful} como a mesma pesquisa, demonstrada nas seções \ref{subsection: pesquisa-lookup-match} e \ref{subsection: pesquisa-manual}, pode ser implementada usando o \textit{Alpha Restful}.
    
\begin{lstlisting}[style=ES6, caption={Busca em Dados Normalizados com o \textit{Alpha Restful}\label{lst: busca-alpha-restful}}]
    let casas = await restful.query({
        "pessoas.casas.pessoas.idade": 40
    }, Casa);
\end{lstlisting}
    
Anteriormente, para a realização dessa pesquisa, foi necessário entre 30 (exemplo \ref{lst: busca-manual}) e 67 (exemplo \ref{lst: busca-lookup-match}) linhas, mas com o \textit{Alpha Restful}, apenas 3 linhas foi o suficiente. Isso ocorre porque o \textit{Alpha Restful} consegue enxergar todos os atributos presentes em outros documentos normalizados, como se eles estivessem dentro do documento de maneira desnormalizada. A sintaxe utilizada para realizar buscas é uma extensão da sintaxe utilizada pelo \textit{Mongoose}, com a diferença de considerar nas pesquisas os atributos contidos em outros documentos relacionados.

\section{Remoção em Cascata de Documentos Relacionados\label{section: remocao-cascata-documentos-relacionados}}
    
No exemplo a qual está sendo abordado, pode-se supor que, por exemplo, o sistema possua uma regra de negócio que afirme que quando uma pessoa for removida, todas as casas pertencentes a ela devam ser removidas também. Para a implementação de tal regra, sem o uso de um \textit{framework}, toda vez que uma pessoa for removida, será necessário manualmente buscar por todas as casas pertencentes a ela e removê-las. O problema desta abordagem manual é que uma coleção pode possuir cada vez mais relacionamentos com outros documentos, deixando o código cada vez mais complexo.

Pensando nisto, o \textit{Alpha Restful} disponibiliza uma opção na sincronização das entidades (\textit{sync}), que implementa exatamente esta funcionalidade. Para garantir este comportamento, é necessário apenas informar a opção \textit{deleteCascade} no atributo da entidade a qual deseja-se que seja removida automaticamente. No exemplo \ref{lst: delete-cascade-alpha-restful} é apresentada tal implementação.

\begin{lstlisting}[style=ES6, caption={Modelagem de ``Casa'' com \textit{deleteCascade}\label{lst: delete-cascade-alpha-restful}}]
    const Casa = new Entity({
        name: "Casa",
        // ...
        sync: {
            pessoas: {
                name: "Pessoa",
                syncronized: ["casas"]
                fill: true,
                jsonIgnoreProperties: "casas",
                deleteCascade: true
            },
            // ...
        }
    })
\end{lstlisting}

\section{Relação de Dependência Entre os Documentos\label{section: relacao-dependencia-entre-documentos}}

No exemplo a qual está sendo abordado, se uma casa não puder ser removida, caso possua um relacionamento com alguma pessoa, seria necessário a verificação da existência de tal relacionamento, antes de uma casa ser removida. Caso todo esse procedimento seja feito manualmente e outras entidades comecem a se relacionar com a entidade Casa, o código desta verificação ficaria cada vez mais complexo, se esta regra se repetisse para outros documentos.

Para que esta regra seja aplicada de forma mais simples, o \textit{Alpha Restful} disponibiliza no ``\textit{sync}'' uma opção que define uma relação de dependência entre os documentos. Uma relação de dependência entre documentos normalizados relacionados garante que um documento não possa ser removido se estiver presente em algum relacionamento definido como dependente. Se, por exemplo, uma casa não puder ser removida, caso possua alguma pessoa, bastaria apenas informar a opção \textit{required} no atributo da entidade a qual deseja-se criar um relacionamento de dependência (Pessoa). O exemplo \ref{lst: required-alpha-restful} apresenta a codificação necessária para que tal opção seja utilizada.

\begin{lstlisting}[style=ES6, caption={Modelagem de Pessoa com \textit{required}\label{lst: required-alpha-restful}}]
    const Pessoa = new Entity({
        name: "Pessoa",
        // ...
        sync: {
            casas: {
                name: "Casa",
                fill: true,
                jsonIgnoreProperties: "pessoas",
                required: true
            },
            // ...
        }
    })
\end{lstlisting}

\section{Remoção de Identificadores Apontando para Lixo\label{section: identificadores-apontando-para-lixo}}

Um dos possíveis problemas que podem ser comuns no desenvolvimento de uma aplicação com \textit{MongoDB}, é a existência de identificadores que apontam para documentos que não existem. Isto acontece porque as entidades que possuem um identificador de outra entidade que foi removida do banco, podem continuar com esse identificador. Um exemplo que pode ser apresentado é de que se uma casa for removida, isso pode fazer com que o identificador dessa casa nos documentos da coleção de pessoas apontará para lugar algum, pois a casa a qual tais identificadores apontam, já não existe mais.
    
Para que este problema seja contornado, sem o uso de um \textit{framework}, é necessário que antes que qualquer entidade seja removida, seja realizado uma análise em todas as entidades que se relacionam com a instância a qual deseja-se remover. Após esta análise, os dados que apontariam para esta entidade que está sendo removida seriam removidos também. Com o aumento da complexidade da aplicação, esse código ficaria cada vez mais e mais complexo, pois a cada novo relacionamento entre documentos, mais alterações precisariam ser feitas no código para garantir esse comportamento. O \textit{Alpha Restful} já resolve esse problema automaticamente (bastando apenas realizar o procedimento descrito na seção \ref{section: delete-sync}). Nenhuma opção precisa ser habilitada na modelagem para que esse problema seja mitigado.

\section{\textit{deleteSync\label{section: delete-sync}}}

Para que as funcionalidades relacionadas à remoção de entidades no \textit{Alpha Restful} possam acontecer (seções \ref{section: remocao-cascata-documentos-relacionados}, \ref{section: relacao-dependencia-entre-documentos} e \ref{section: identificadores-apontando-para-lixo}), é necessário que o método \textit{restful.deleteSync} seja chamado antes que qualquer entidade ser removida. O exemplo \ref{lst: delete-sync-alpha-restful} apresenta o código a ser executado antes de qualquer casas ser removida.

\begin{lstlisting}[style=ES6, caption={Antes de Remover Uma Casa\label{lst: delete-sync-alpha-restful}}]
  await restful.deleteSync(casa._id, "Casa", Casa.syncronized)
\end{lstlisting}
\chapter{CONCLUSÃO}
\label{Conclusao}

Os bancos de dados NoSQL trazem o conceito de que o desenvolvimento de uma aplicação não precisa necessariamente estar preso às regras e limites do modelo relacional. A proposta de tais bancos inclui mudar a estrutura de como os dados são armazenados, a fim de eliminar alguma barreira imposta pelo SQL. No caso do \textit{MongoDB}, propõe-se otimizar o armazenamento e gerenciamento de dados que não precisam seguir as formas normais. Ao mesmo tempo, ele não impede a normalização, onde tal abordagem for pertinente e se demonstrar uma vantagem. Permitir que a aplicação decida sobre a normalização ou desnormalização, demonstra-se ser um bom caminho, pois possibilita que a melhor decisão possa ser tomada, dependendo das necessidades da aplicação a ser desenvolvida.

O \textit{MongoDB} disponibiliza, de forma oficial, várias recursos para auxiliar na desnormalização dos dados. Apesar disso, este trabalho demonstrou que, até mesmo em aplicações onde a desnormalização é importante, eventualmente, os dados poderão precisar estar normalizados. Quando isso acontecer, várias operações poderão ser bastante complexas de serem desenvolvidas, em comparação com o SQL. Nesse cenário, este trabalho apresentou o \textit{Alpha Restful} como uma solução capaz de minimizar tal problemática, simplificando operações complexas e disponibilizando novos recursos para elas.

%%%%% REMOVIDO %%%%%
% O \textit{Alpha Restful} é um \textit{framework} que tem uma proposta ousada. Um de seus principais objetivos de seu desenvolvimento foi fazer com que a escolha de se normalizar ou desnormalizar os dados no \textit{MongoDB}, apareça mais como uma questão de "regra de negócio", do que como uma questão estrutural profunda na forma como entidades são modeladas e buscadas. Sendo essa finalidade alcançada, a facilidade que já existe para se desnormalizar os dados no \textit{MongoDB}, também existiria para normalizá-los. Isso eliminaria um dos obstáculos que atualmente encontram-se na utilização do banco de dados: a normalização de dados do \textit{MongoDB} que, como foi demonstrado anteriormente, as vezes precisa ser feita em, pelo menos, parte dos dados armazenados, pode ser mais complexo que na desnormalização.

% Como foi demonstrado, esse objetivo foi atingido parcialmente\footnote{Parcilmente?} na versão atualmente feita do \textit{framework} (0.7.37). As pesquisas normalizadas usando o \textit{Alpha Restful}, são tão simples de serem realizadas quanto aquelas que acessam dados desnormalizados. A união de documentos pode ser feita de forma simples, com um recurso não disponíveis nas outras ferramentas utilizadas: o relacionamento inverso.
%%%%% REMOVIDO %%%%%

A fim de mostrar as vantagens do \textit{Alpha Restful}, cinco funcionalidades comuns para o desenvolvimento de aplicações com \textit{MongoDB} foram analisadas. Tais análises demonstram a relevância da solução proposta para melhorar o processo de desenvolvimento de tais funcionalidades. Com o \textit{Alpha Restful}, as pesquisas normalizadas demonstraram ser mais simples que os disponíveis em outras ferramentas de mercado. Além disso, nas junções de documentos, o \textit{framework} proposto disponibiliza dois novos recursos (o relacionamento inverso e o relacionamento transitivo). Com esses novos recursos, novas funcionalidades podem ser implementadas, de maneira simples e intuitiva.

A terceira, quarta e quinta funcionalidades são tratadas pelo \textit{framework} de maneira automática, bastando ativá-las através do uso de opções no objeto de sincronização do atributo. Isso é uma grande melhoria, pois nas outras ferramentas de mercado apresentadas, essas funcionalidades não são trabalhadas automaticamente, exigindo implementações manuais mais complexas de se fazer.

Uma das coisas que precisa ficar claras sobre o \textit{Alpha Restful} é de que ele está em verão \textit{beta}. A sua atual implementação é apenas uma prova de conceito. O objetivo de tal prova é propor uma nova ferramenta funcional para suprir as necessidades que estão descritas nesse trabalho.

As funcionalidades do \textit{Alpha Restful} aqui descritas, já estão disponíveis e foram testadas. O motivo se denominar essa ferramenta em fase \textit{beta} e como uma prova de conceito, é de que não foram feitas as otimizações necessárias para que uma aplicação que utilize tal \textit{framework} possa ser utilizada em produção. Além desse fato, algumas funcionalidades também relevantes não foram desenvolvidas ainda.

No futuro, pretende-se permitir a definição de sub-consultas dentro da própria modelagem das entidades. Isto é equivalente às \textit{views} do SQL. Além disso, sub-consultas poderão ser utilizadas nos métodos de busca descritos na seção \ref{subsection: pesquisa-usando-alpha-restful}. Pretende-se também, disponibilizar interfaces para que outros bancos de dados NoSQL possam ser usados pelo \textit{Alpha Restful}. Outras melhorias de performance serão realizadas a fim de que uma versão estável possa ser lançada.

%%%%% REMOVIDO %%%%%
% Depois que a versão atual do \textit{Alpha Restful} foi desenvolvida, foi observado que o \textit{framework} pode ser melhor otimizado e que pode propor algo mais ousado. Por causa disso, o \textit{Alpha Restful} será completamente refeito. Isso será necessário para que uma nova proposta de desenvolvimento possa ser feita.
%%%%% REMOVIDO %%%%%

% ELEMENTOS PÓS-TEXTUAIS
\postextual
% Referências bibliográficas
\bibliography{documento} 
\include{outrasFolhas/Apendice} 
\include{outrasFolhas/Anexos} 


% INDICE REMISSIVO
%\printindex

\end{document}

