\vfill
\begin{center}
{\textbf{RESUMO}\\}
\end{center}
\noindent

Com a crescente demanda para armazenar uma quantidade cada vez maior de dados, os modelos de banco de dados tradicionais vêm demonstrando dificuldades no desenvolvimento de aplicações escaláveis, disponíveis e consistentes. Por isso, o modelo NoSQL tornou-se uma tecnologia emergente para suprir as deficiências do modelo relacional. Dentre os banco de dados NoSQL mais populares, encontra-se o \textit{MongoDB} que, apesar de facilitar o armazenamento de dados desnormalizados, muitas vezes não possui recursos simples e intuitivos para a manipulação de dados normalizados. Neste contexto, este trabalho apresenta um \textit{framework}, denominado \textit{Alpha Restful}, que propõe facilitar o desenvolvimento de funcionalidades sobre dados normalizados utilizando o \textit{MongoDB}. Como resultado, através de exemplos de código fonte, demonstrou-se de que forma o \textit{Alpha Restful} se sobressai em relação às alternativas apresentadas, comprovando a relevância deste estudo.

\vspace{\onelineskip}
 \noindent
 \textbf{Palavras-chaves}: \textit{MongoDB}; Normalização; \textit{Framework}.
