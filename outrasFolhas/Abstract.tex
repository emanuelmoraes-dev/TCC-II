\vfill
\begin{center}
{\textbf{ABSTRACT}\\}
\end{center}

\noindent

With the growing demand for storing an increasing amount of data, traditional database models have shown difficulties in developing scalable, available and consistent applications. For this reason, the NoSQL model has become an emerging technology to supply deficiencies in the relational model. Among the most popular NoSQL databases, MongoDB is found that, despite facilitating the storage of unnormalized data, it often does not have simple and intuitive resources for handling normalized data. In this context, this work presents a framework, called Alpha Restful, which proposes to facilitate the development of features on normalized data using MongoDB. As a result, through examples of source code, it was demonstrated how Alpha Restful stands out in relation to the alternatives presented, proving the relevance of this study.
 
 \vspace{\onelineskip}
    
 \noindent
 \textbf{Key words}: MongoDB; Normalization; \textit{Framework}; Alpha Restful.
