\chapter{TRABALHOS RELACIONADOS}
\label{TrabalhosRelacionados}

O \textit{MongoDB} é um dos bancos de dados NoSQL mais populares e utilizados. Isso pode ser demonstrado pela grande quantidade de citações a ele na literatura\hl{, além da pesquisa feita pela \textit{DB-Engines}}\footnote{A pesquisa e a metodologia da pesquisa estão respectivamente disponíveis pelos \textit{links} \url{https://db-engines.com/} e \url{https://db-engines.com/en/ranking_definition}. Acessos ocorridos no dia 16 de dezembro de 2020, às 16:18.} \hl{citá-lo como sendo o banco de dados NoSQL mais popular em novembro de 2020.} Vários são os trabalhos que o comparam a outros bancos de dados relacionais ou não relacionais. Os trabalhos de \citeonline{rautmare2016mysql}; \citeonline{seo2017performance}; \citeonline{grover2016mvm}; \citeonline{fatima2016comparison}; e \citeonline{ansari2018performance} fazem uma comparação de desempenho do \textit{MongoDB} com o banco de dados relacional \textit{MySql}. Neles são gerados gráficos e tabelas que medem o tempo de resposta de cada banco de dados, pela quantidade de dados teste usados em operações de leitura e escrita. \hl{ } Tais resultados demonstram um ganho significativo de performance do \textit{MongoDB} em vários casos de uso. Isto demonstra a relevância da ferramenta, principalmente em aplicações com grande quantidade de dados (\textit{Big Data}). Dentre os trabalhos apresentados, \citeonline{fatima2016comparison} realizam testes com dados reais, visando simular um ambiente IoT, na qual existe a captura de uma grande quantidade de dados de sensores. \hl{Os testes realizados também fazem um comparativo} com o banco de dados \textit{VoltBD}, que utiliza um novo modelo de banco de dados chamado de \textit{NewSQL}.

Além \hl{dos trabalhos apresentados}, também pode ser citado o trabalho \hl{de} \citeonline{pandey2020performance}, que realiza um comparativo mais completo que os apresentados anteriormente, analisando \hl{diversos casos} de teste diferentes. Tais testes também foram realizados sobre o \textit{MongoDB} e o \textit{MySql}, porém medindo o tempo de resposta por quantidade de \textit{threads} em execução, \textit{throughput} (operações/segundo) e latência média, em operações de leitura e escrita. Os resultados também demonstram uma vantagem do \textit{MongoDB} em vários cenários. Dentre os trabalhos apresentados anteriormente, comparativos \hl{que envolvem} cenários \hl{com} várias \textit{threads} ou \hl{a realização de operações simultâneas e distintas} também podem ser encontrados em \citeonline{rautmare2016mysql}; \citeonline{fatima2016comparison}; e \citeonline{ansari2018performance}.

Outros trabalhos que realizam um comparativo com outros bancos de dados podem ser encontrados. \citeonline{moreno2016performance} fazem um comparativo com o banco de dados \textit{Oracle}. Já \citeonline{chopade2017mongodb} realizam um comparativo com outro banco de dados NoSQL chamado de \textit{CouchBase}, utilizando dados de imagens como teste.

As afirmativas feitas na introdução deste trabalho sobre a dificuldade na implementação de operações sobre dados normalizados no \textit{MongoDB} em comparação com o SQL, são confirmadas \hl{pelo trabalho desenvolvido por} \citeonline{celesti2019study}. Neste trabalho, \hl{implementações} manual da junção de documentos, são realizadas usando o \textit{MongoDB}. \hl{Nele, a junção de documentos no \textit{MongoDB}, por meio de uma operação equivalente ao \textit{inner join} do SQL, tornou-se, não só mais complexo de ser desenvolvido, como apresentou performance inferior em comparação com um banco relacional}. Tal resultado demonstra a necessidade e importância de \textit{APIs} e \textit{frameworks} tratarem tal operação, a fim de se obter \hl{ganhos de performance} e facilidade no desenvolvimento. Na seção \ref{section: juncao-documentos}, tal operação de junção será explicitada e alternativas à implementação manual feita por \citeonline{celesti2019study} serão demonstradas, por meio do uso de ferramentas externas ao \textit{MongoDB}.

O \textit{survey} de \citeonline{davoudian2018survey}, assim como os trabalhos apresentados, confirmam as afirmativas deste trabalho da importância do NoSQL na aplicação e desenvolvimento de diversas aplicações, cujo o uso do modelo tradicional Relacional deixa a desejar. Em \citeonline{davoudian2018survey} podem ser encontradas mais informações e detalhes sobre o uso de bancos de \textit{NoSQL}, principalmente no escopo de aplicações distribuídas e escaláveis de forma horizontal.

\hl{O \textit{Mongoose}, sendo uma biblioteca que permite a criação de \textit{schemas} no \textit{MongoDB}, foi escolhido para ser alvo de comparação neste trabalho por causa de sua grande relevância, tando na literatura, quanto na sua popularidade no desenvolvimento de aplicações com \textit{MongoDB} em \textit{Node JS}. O quadro} \ref{quad: livros_mongoose} \hl{referencia vários livros que citam o \textit{Mongoose}, demonstrando sua relevância para o desenvolvimento de aplicações com \textit{MongoDB}.}

% \begin{table}[h]
% \caption{Livros que citam o \textit{Mongoose} \label{livros_mongoose}}
% \begin{tabular}{@{}|l|l|@{}}
% \toprule
% \textbf{Título}                                  & \textbf{Autores}                 \\ \midrule
% \textit{MERN Quick Start Guide}                           & \citeonline{wilson2018mern}      \\ \midrule
% \textit{Web development with MongoDB and NodeJs}          & \citeonline{satheesh2015web}     \\ \midrule
% \textit{Building Node. js REST API with TDD Approach}     & \citeonline{pandian2018building} \\ \midrule
% \textit{Full Stack JavaScript}                            & \citeonline{mardan2018intro}     \\ \midrule
% \textit{Mongoose for Application Development}             & \citeonline{holmes2013mongoose}  \\ \midrule
% \textit{Node. js, MongoDB, and AngularJS web development} & \citeonline{dayley2014node}      \\ \midrule
% \textit{JavaScript Frameworks for Modern Web Development} & \citeonline{bin2019mongoose}     \\ \bottomrule
% \end{tabular}
% \end{table}

\begin{quadro}[h]
\caption{Livros que citam o \textit{Mongoose} \label{quad: livros_mongoose}}
\begin{tabular}{|l|l|}
\hline
\def\arraystretch{1.5}
\begin{tabular}[c]{@{}l@{}}\small{\textbf{Título}}\end{tabular} & \begin{tabular}[c]{@{}l@{}}\small{\textbf{Autores}}\end{tabular} \\ \hline
\def\arraystretch{1.5}
\begin{tabular}[c]{@{}l@{}}\small{MERN Quick Start Guide}\end{tabular} & \small{\citeonline{wilson2018mern}} \\ \hline
\def\arraystretch{1.5}
\begin{tabular}[c]{@{}l@{}}\small{Web development with MongoDB and NodeJs}\end{tabular} & \small{\citeonline{satheesh2015web}} \\ \hline
\def\arraystretch{1.5}
\begin{tabular}[c]{@{}l@{}}\small{Building Node. js REST API with TDD Approach}\end{tabular} & \small{\citeonline{pandian2018building}} \\ \hline
\def\arraystretch{1.5}
\begin{tabular}[c]{@{}l@{}}\small{Full Stack JavaScript}\end{tabular} & \small{\citeonline{mardan2018intro}} \\ \hline
\def\arraystretch{1.5}
\begin{tabular}[c]{@{}l@{}}\small{Mongoose for Application Development}\end{tabular} & \small{\citeonline{holmes2013mongoose}} \\ \hline
\def\arraystretch{1.5}
\begin{tabular}[c]{@{}l@{}}\small{Node. js, MongoDB, and AngularJS web development}\end{tabular} & \small{\citeonline{dayley2014node}} \\ \hline
\def\arraystretch{1.5}
\begin{tabular}[c]{@{}l@{}}\small{JavaScript Frameworks for Modern Web Development}\end{tabular} & \small{\citeonline{bin2019mongoose}} \\ \hline
\end{tabular}
\end{quadro}

% \begin{longtable}[]{|l|l|}
% \caption{Livros que citam o \textit{Mongoose} \label{livros_mongoose}}\tabularnewline
% \toprule
% \textbf{Título} & \textbf{Autores}\tabularnewline
% \midrule
% \endfirsthead
% \toprule
% \textbf{Título} & \textbf{Autores}\tabularnewline
% \midrule
% \endhead
% MERN Quick Start Guide & \citeonline{wilson2018mern}\tabularnewline\hline%\addlinespace
% Web development with MongoDB and NodeJs & \citeonline{satheesh2015web}\tabularnewline\hline%\addlinespace
% Building Node. js REST API with TDD Approach & \citeonline{pandian2018building}\tabularnewline\hline%\addlinespace
% Full Stack JavaScript & \citeonline{mardan2018intro}\tabularnewline\hline%\addlinespace
% Mongoose for Application Development & \citeonline{holmes2013mongoose}\tabularnewline\hline%\addlinespace
% Node. js, MongoDB, and AngularJS web development & \citeonline{dayley2014node}\tabularnewline\hline%\addlinespace
% JavaScript Frameworks for Modern Web Development & \citeonline{bin2019mongoose}\tabularnewline
% \bottomrule
% % \caption*{Fonte: O autor (2020)}
% \end{longtable}