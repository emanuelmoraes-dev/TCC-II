\chapter{CONCLUSÃO}
\label{Conclusão}

Os bancos de dados NoSQL trazem o conceito de o desenvolvimento de uma aplicação não precisar necessariamente estar preso as regras e limites do modelo relacional. A proposta de tais bancos inclui mudar a estrutura como os dados são armazenados, a fim de eliminar alguma barreira imposta pelo SQL. Cada tipo de banco de dados NoSQL propõe a eliminação de algum limite em específico. No caso do \textit{MongoDB}, propõe-se otimizar o armazenamento e gerenciamento de dados que não precisem seguir as formas normais, ao mesmo tempo que não impede a normalização dos dados, onde tal abordagem for pertinente e se demonstrar uma vantagem. Permitir que a aplicação decida sobre a normalização ou desnormalização dos dados demonstra-se ser um bom caminho, pois possibilita que a melhor decisão possa ser tomada, dependendo das necessidades da aplicação a ser desenvolvida.

O \textit{Alpha Restful} é um framework que tem uma proposta ousada. Um de seus principais objetivos é de fazer com que a escolha de se normalizar ou desnormalizar os dados no \textit{MongoDB}, seja mais como uma questão de regra de negócio, do que como uma questão estrutural profunda na forma como entidades são modeladas e buscadas. Se essa proposta fosse alcançada, a facilidade que já existe para se desnormalizar os dados no \textit{MongoDB}, também existiria para normalizá-los. Isso eliminaria um dos obstáculos que atualmente encontram-se na utilização do banco de dados: a normalização de dados do \textit{MongoDB} que, como foi demonstrado anteriormente, as vezes precisa ser feita em, pelo menos, parte dos dados armazenados, pode ser mais complexo que na desnormalização.

Como foi demonstrado, esse objetivo pôde ser atingido parcialmente na versão atualmente feita do \textit{framework} (0.7.37). As pesquisas normalizadas usando o \textit{Alpha Restful}, são tão simples de serem realizadas quanto aquelas que acessam dados desnormalizados. A união de documentos pode ser feita de forma simples, com um recurso não disponíveis nas outras ferramentas utilizadas: o relacionamento inverso.

Uma das coisas que precisam ficar claras sobre esse \textit{framework} é de que ele não está completo. A versão atualmente feita está em versão \textit{beta}. A sua atual implementação é apenas uma prova de conceito. O objetivo de tal prova é propor uma nova ferramenta funcional para suprir as necessidades que estão descritas nesse trabalho.

As funcionalidades do \textit{Alpha Restful} aqui descritas, já estão funcionando e foram muito bem testadas. O motivo se denomiar essa ferramenta em fase \textit{beta} e como uma prova de conceito, é de que não foram feitas as otimizações necessárias para que uma aplicação que utilize tal \textit{framework} possa ser utilizada em produção. Além desse fato, algumas funcionalidades importantes não poderam ser desenvolvidas ainda.

% Depois que a versão atual do \textit{Alpha Restful} foi desenvolvida, foi observado que o \textit{framework} pode ser melhor otimizado e que pode propor algo mais ousado. Por causa disso, o \textit{Alpha Restful} será completamente refeito. Isso será necessário para que uma nova proposta de desenvolvimento possa ser feita.

Dentre todas as funcionalidades do \textit{framework}, as 5 funcionalidades apresentadas nesse trabalho (seções 3.1, 3.2, 3.3, 3.4 e 3.5) são as mais relevantes para o uso do \textit{MongoDB}. Outras funcionalidades também foram implementadas, porém seu escopo de atuação está mais intimamente relacionado à web. Como o escopo desse trabalho é fazer uma análise mais próxima do \textit{MongoDB}, tais funcionalidades foram omitidas nesse documento. Apesar disso, quase todas as funcionalidades desenvolvidas estão disponíveis em um guia de uso gratuito e online, disponiblizado pelo \textit{link} https://www.npmjs.com/package/alpha-restful. Por meio desse mesmo \textit{link}, é possível baixar a ferramenta e utilizá-la. Um dos resultados relevantes que pode ser encontrado nesse link é de que até agora (24/03/2020), mais de 6000 \textit{downloads} foram feitos, desde o seu lençamento na internet (24/02/2019).