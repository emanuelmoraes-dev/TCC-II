\chapter{CONCLUSÃO}
\label{Conclusao}

Os bancos de dados NoSQL trazem o conceito de que o desenvolvimento de uma aplicação não precisa necessariamente estar preso às regras e limites do modelo relacional. A proposta de tais bancos inclui mudar a estrutura de como os dados são armazenados, a fim de eliminar alguma barreira imposta pelo SQL. No caso do \textit{MongoDB}, propõe-se otimizar o armazenamento e gerenciamento de dados que não precisam seguir as formas normais. Ao mesmo tempo, ele não impede a normalização, onde tal abordagem for pertinente e se demonstrar uma vantagem. Permitir que a aplicação decida sobre a normalização ou desnormalização, demonstra-se ser um bom caminho, pois possibilita que a melhor decisão possa ser tomada, dependendo das necessidades da aplicação a ser desenvolvida.

O \textit{MongoDB} disponibiliza, de forma oficial, várias recursos para auxiliar na desnormalização dos dados. Apesar disso, este trabalho demonstrou que, até mesmo em aplicações onde a desnormalização é importante, eventualmente, os dados poderão precisar estar normalizados. Quando isso acontecer, várias operações poderão ser bastante complexas de serem desenvolvidas, em comparação com o SQL. Nesse cenário, este trabalho apresentou o \textit{Alpha Restful} como uma solução capaz de minimizar tal problemática, simplificando operações complexas e disponibilizando novos recursos para elas.

%%%%% REMOVIDO %%%%%
% O \textit{Alpha Restful} é um \textit{framework} que tem uma proposta ousada. Um de seus principais objetivos de seu desenvolvimento foi fazer com que a escolha de se normalizar ou desnormalizar os dados no \textit{MongoDB}, apareça mais como uma questão de "regra de negócio", do que como uma questão estrutural profunda na forma como entidades são modeladas e buscadas. Sendo essa finalidade alcançada, a facilidade que já existe para se desnormalizar os dados no \textit{MongoDB}, também existiria para normalizá-los. Isso eliminaria um dos obstáculos que atualmente encontram-se na utilização do banco de dados: a normalização de dados do \textit{MongoDB} que, como foi demonstrado anteriormente, as vezes precisa ser feita em, pelo menos, parte dos dados armazenados, pode ser mais complexo que na desnormalização.

% Como foi demonstrado, esse objetivo foi atingido parcialmente\footnote{Parcilmente?} na versão atualmente feita do \textit{framework} (0.7.37). As pesquisas normalizadas usando o \textit{Alpha Restful}, são tão simples de serem realizadas quanto aquelas que acessam dados desnormalizados. A união de documentos pode ser feita de forma simples, com um recurso não disponíveis nas outras ferramentas utilizadas: o relacionamento inverso.
%%%%% REMOVIDO %%%%%

A fim de mostrar as vantagens do \textit{Alpha Restful}, cinco funcionalidades comuns para o desenvolvimento de aplicações com \textit{MongoDB} foram analisadas. Tais análises demonstram a relevância da solução proposta para melhorar o processo de desenvolvimento de tais funcionalidades. Com o \textit{Alpha Restful}, as pesquisas normalizadas demonstraram ser mais simples que os disponíveis em outras ferramentas de mercado. Além disso, nas junções de documentos, o \textit{framework} proposto disponibiliza dois novos recursos (o relacionamento inverso e o relacionamento transitivo). Com esses novos recursos, novas funcionalidades podem ser implementadas, de maneira simples e intuitiva.

A terceira, quarta e quinta funcionalidades são tratadas pelo \textit{framework} de maneira automática, bastando ativá-las através do uso de opções no objeto de sincronização do atributo. Isso é uma grande melhoria, pois nas outras ferramentas de mercado apresentadas, essas funcionalidades não são trabalhadas automaticamente, exigindo implementações manuais mais complexas de se fazer.

Uma das coisas que precisa ficar claras sobre o \textit{Alpha Restful} é de que ele está em verão \textit{beta}. A sua atual implementação é apenas uma prova de conceito. O objetivo de tal prova é propor uma nova ferramenta funcional para suprir as necessidades que estão descritas nesse trabalho.

As funcionalidades do \textit{Alpha Restful} aqui descritas, já estão disponíveis e foram testadas. O motivo se denominar essa ferramenta em fase \textit{beta} e como uma prova de conceito, é de que não foram feitas as otimizações necessárias para que uma aplicação que utilize tal \textit{framework} possa ser utilizada em produção. Além desse fato, algumas funcionalidades também relevantes não foram desenvolvidas ainda.

No futuro, pretende-se permitir a definição de sub-consultas dentro da própria modelagem das entidades. Isto é equivalente às \textit{views} do SQL. Além disso, sub-consultas poderão ser utilizadas nos métodos de busca descritos na seção \ref{subsection: pesquisa-usando-alpha-restful}. Pretende-se também, disponibilizar interfaces para que outros bancos de dados NoSQL possam ser usados pelo \textit{Alpha Restful}. Outras melhorias de performance serão realizadas a fim de que uma versão estável possa ser lançada.

%%%%% REMOVIDO %%%%%
% Depois que a versão atual do \textit{Alpha Restful} foi desenvolvida, foi observado que o \textit{framework} pode ser melhor otimizado e que pode propor algo mais ousado. Por causa disso, o \textit{Alpha Restful} será completamente refeito. Isso será necessário para que uma nova proposta de desenvolvimento possa ser feita.
%%%%% REMOVIDO %%%%%