\chapter{\textit{METODOLOGIA}}
\label{Metodologia}

Assim como descrito anteriormente, o foco do \textit{MongoDB} é o armazenamento de dados desnormalizados, disponibilizando diversas funcionalidades para a sua manipulação. Por essa razão, operações de busca e união de documentos normalizados precisam, as vezes, serem implementadas de forma menos intuitiva e mais trabalhosa, em comparação com o SQL. Visando amenizar os problemas apresentados, foi desenvolvido um framework denominado de \textit{Alpha Restful}. Tal ferramenta foi projetada para a linguagem \textit{JavaScript}, usando o ambiente de execução \textit{Node JS}, utilizando internamente o \textit{MongoDB} e o \textit{Mongoose} (uma biblioteca que implementa algumas funcionalidades extras para o \textit{MongoDB}).

O \textit{Alpha Restful} facilita e automatiza diversas funcionalidades para o desenvolvimento de aplicações WEB com \textit{MongoDB}. O presente trabalho tem a proposta de apresentar este framework como uma solução para alguns problemas que envolvem a manipulação de dados normalizados com o \textit{MongoDB}. Por esta razão, dentre todas as funcionalidades disponíves no framework, foram selecionadas 5 delas (Junção de Documentos, Buscas filtradas em Documentos Relacionados, Remoção em Cascata de Documentos Relacionados, Relação de Dependência Entre os Documentos e Identificadores Apontando para Lixo) que melhoram o processo de desenvolvimento de aplicações, dentro do escopo de manipulação de dados normalizados presentes no banco. Outras funcionalidades, mais relacionadas ao desenvolvimento WEB e não tão relacionadas ao \textit{MongoDB} em sí, serão omitidas neste documento, por fugir do escopo principal proposto.

O \textit{Alpha Restful} implementa tais funcionalidades utilizando um conceito apelidado de ``sincronizaçao'', representado pelo objeto de sincronização (\textit{sync}). A sincronização é o estabelecimento de uma conexão lógica entre um atributo de uma entidade com outra entidade. Uma das formas de se estabelecer essa conexão é através do armazenamento de identificadores de outros documentos. Isso seria equivalente ao uso de \textit{foreign key} no modelo relacional. Outras formas de se estabelecer essa conexão não serão explicitadas neste trabalho. Uma sincronização pode ocorrer, tanto na modelagem das entidades, quanto de forma dinâmica, antes ou depois de uma pesquisa.

Uma vez que duas entidades estão sincronizadas, diversas funções podem ser definidas. Cada uma delas acionam eventos que esperam uma das duas entidades serem chamadas por algum método presente no framework. Tanto atributos reais (presentes no banco de dados) quanto atributos dinâmicos (não presentes no banco de dados, pois é criado dinamicamente em memória) podem ser alvo de uma sincronização. Um exemplo de atributo dinâmico será explicitado na seção 4.1.5.1, quando for explicado o funcionamento do relacionamento inverso. Esses atributos dinâmicos são tratados como se eles existissem dentro do banco de dados, sendo possível utilizá-los em pesquisas. Esse tipo de atributo pode ser definido, tanto na modelagem da entidade, quanto dinamicamente na chamada de uma pesquisa ou junção de documentos.

A fim de demonstrar e justificar a importância da ferramenta desenvolvida, bem como explicitar de que forma os problemas apresentados foram solucionados ou amenizados, a seção de ``RESULTADOS'' irá apresentar um comparativo de como cada uma das 5 funcionalidades escolhidas podem ser implementadas com outras ferramentas já disponíveis no mercado, além de demonstrar quais os problemas ou dificuldades são enfrentadas usando tais ferramentas e como a utilização do \textit{Alpha Restful} pode resolver ou amenizar essas problemáticas. A fim de facilitar a apresentação do comparativo, na próxima seção será mostrado um conjunto de documentos exemplo que serão usados como base para explicar como cada funcionalidade pode manipular esses dados.

%diminuir as frases

Para as duas primeiras funcionalidades (Junção de Documentos e Buscas Filtradas em Documentos Relacionados), será explicado, com a demonstração de exemplos práticos do código fonte, como outras ferramentas, presentes no \textit{MongoDB} ou no \textit{Mongoose}, podem ser utilizadas, bem como as vantagens e desvantagens de cada abordagem, demonstrando quais problemas são persistidos em cada uma delas. Após essa demonstração, será apresentado (também com a utilização de exemplos práticos de código fonte) como essa mesma função pode ser feita utilizando o \textit{Alpha Restful}, mostrando como o framework resolve ou ameniza os problemas apresentados anteriormente, bem como quais opções a mais estão disponíveis por meio dessa ferramenta. Para as demais funcionalidades, é apresentado uma breve descrição de como elas podem manualmente ser implementadas usando o \textit{MongoDB} e logo após é apresentado como é possível, através do \textit{Alpha Restful}, resolver tal necessidade de maneira mais automatizada.